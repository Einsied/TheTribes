\documentclass[a4paper]{book}

\usepackage[english]{babel}
\usepackage[utf8]{inputenc}
\usepackage[T1]{fontenc}
\usepackage{lmodern}

\newcommand{\Author}{Sebastian Einsiedler}
\newcommand{\Title}{Settlers or cultures like game}

\title{\Title}
\author{\Author}

\usepackage{amsmath, amssymb}
\usepackage{enumerate}
\usepackage[table]{xcolor}
\usepackage[singlespacing]{setspace}
\usepackage{hyperref}
\hypersetup{
	bookmarks=true,
	unicode=true,
	pdfauthor = {\Author},
	pdftitle = {\Title},
	pdfsubject = {\Title},
	pdfsubject = {\Author, \Title, Game development},
	colorlinks = true,
	linkcolor = {blue!50!black},
	filecolor = {red!50!black},
	urlcolor = {blue!75!black}
}
\usepackage{csquotes}

\usepackage{footnotebackref}
\usepackage{longtable}
\usepackage{booktabs}
\usepackage{graphicx}
\usepackage[toc, section=section]{glossaries}
\usepackage{pdflscape}
\usepackage{tikz}
\usetikzlibrary{positioning}
\tikzset{
	% The raw resources that renew like game or trees
	raw_renewable/.style={
		draw=green,rectangle
	},
	% The raw resources that are limited like ore or stone
	raw_limited/.style={
		draw=gray,rectangle
	},
	% The livestock
	livestock/.style={
		draw=red,rectangle
	},
	% The resources that are only obtained by farming
	produce/.style={
		draw=yellow,rectangle
	},
	% The intermediate products
	intermediate/.style={
		draw=blue,rectangle
	},
	% The building material
	material/.style={
		draw=black,rectangle
	},
	% The finished foods
	food/.style={
		draw=green!50!yellow,rectangle
	},
	% Normal goods
	good/.style={
		draw=purple,rectangle
	},
	% Normal goods
	tool/.style={
		draw=purple!50,rectangle
	},
	% The luxury goods
	luxury/.style={
		draw=yellow!50,rectangle
	},
	% Weapons
	weapon/.style={
		draw=red!50,rectangle
	},
}


\newglossary*{tribe}{Tribes}
\newglossaryentry{Vikings}{
	name=children of the spirits,
	type=tribe,
	description={simple nature loving craftsmen,
		with a strong connection to the land and nature that surrounds them}
}
\newglossaryentry{Nomads}{
	name=mounted men,
	type=tribe,
	description={fierce warriors and raiders, that raise large herds and conquer their foes
		on the back of their mounts with their cruel gods and the wind in their back}
}
\newglossaryentry{Egyptians}{
	name=river folk,
	type=tribe,
	description={erect great monuments within fertile river plains.
	They fight for the glory of their kings and an immortal afterlife.}
}
\newglossaryentry{Aztecs}{
	name=servants of the serpent,
	type=tribe,
	description={need to feed the sky serpent with human sacrifice so it can then be
		consumed by sun.
		Born from corn and water they are constantly on the search for new sacrifices.
		For the serpent hungers and if you do not feed it, it will consume you}
}
\newglossaryentry{Japanese}{
	name=empire,
	type=tribe,
	description={lies within the fertile plains-
		Its peasants grow rice, its craftsman forge tools and weapons,
		so its priest and artists can create great works for the celestial dragons.
		Its warriors defend the borders with bow, katana and musket against all invaders}
}
\newglossaryentry{Romans}{
	name=republics,
	type=tribe,
	description={stand against the gods.
	Where other chose slavery in the face of the supernatural the republics chose industry and science.
	Heavily armored its legions advance towards its enemy,
	while the mortars turn fertile plains into burning hell.
	The complex bureaucracy and trade deals enable the republics to field
	powerful legions commanded by officers and supported by fearsome war-machines,
	to lay low all that oppose the might of humanity}
}

\newglossary*{good}{Goods}

% Abstract terms, like weapon, armor or tool
\newglossaryentry{Food}{
	name=food,
	type=good,
	description={anything the members of the \hyperref[ch:Tribes]{tribes} can consume for sustenance.
			\hyperref[ch:Goods:Food]{Food}-goods have different quality and nutritional values.
			Since a lot of ingredients are edible they are also
			\hyperref[ch:Goods:Food]{food}-goods}
}

\newglossaryentry{Merchandise}{
	name=merchandise,
	type=good,
	description={anything the members of the \hyperref[ch:Tribes]{tribes} consume for joy.
			\hyperref[ch:Goods:Merchandise]{Merchandise} usually comes with a quality level and durability}
}

\newglossaryentry{Equipment}{
	name=equipment,
	type=good,
	description={anything the members of the \hyperref[ch:Tribes]{tribes} use to enhance their themselves
			\Glspl{Shoe} are a great example for \hyperref[ch:Goods:Equipment]{equipment} permitting faster walking}
}

\newglossaryentry{Tool}{
	name=tool,
	type=good,
	description={anything the members of the \hyperref[ch:Tribes]{tribes} use to do their work except for \glspl{Weapon}.
			\Glspl{Pick} are a great example for \hyperref[ch:Goods:Tools]{tools} often used in mining}
}

\newglossaryentry{Weapon}{
	name=weapon,
	type=good,
	description={is an implements of war.
			\hyperref[ch:Goods:Armory:Weapons]{Weapons} are wielded by the members of the \hyperref[ch:Tribes]{tribes} against each other
			and the different \hyperref[ch:World:Inhabitants]{animals}}
}

\newglossaryentry{Armor}{
	name=armor,
	type=good,
	description={is an implements of war.
			\hyperref[ch:Goods:Armory:Weapons]{Armor} is used by the members of the \hyperref[ch:Tribes]{tribes} to protect themselves
			form each other and \hyperref[ch:World:Inhabitants:Animals]{wild animals}}
}

\newglossaryentry{Medicine}{
	name=medicine,
	type=good,
	description={is used to cure diseases or wounds sustained in \hyperref[ch:Conflict:Combat]{combat}.
		\hyperref[ch:Goods:Medicine]{Medicine} differs in quality and used ingredients among the \hyperref[ch:Tribes]{tribes}}
}

\newglossaryentry{Drink}{
	name=drink,
	type=good,
	description={anything the members of the \hyperref[ch:Tribes]{tribes} can consume for entertainment.
		\hyperref[ch:Goods:Drink]{Drinks} are also used to restore a depleted morale after battles.
		Their make and quality differs among the different \hyperref[ch:Tribes]{Tribes}}
}


% Raw inputs
\newglossaryentry{Game}{
	name=game,
	type=good,
	description={refers to wild \hyperref[ch:Goods:Nature:Animals]{animals} that roam around and can be hunted}
}

\newglossaryentry{Hide}{
	name=hide,
	type=good,
	description={gained by slaugthering \hyperref[ch:World:Inhabitants:Animals]{wild animals} and
			\hyperref[ch:World:Inhabitants:Livestock]{livestock}}
}

\newglossaryentry{Herb}{
	name=herb,
	type=good,
	description={can be harvested from \hyperref[ch:Goods:Nature:Plants:Limited]{plants}}
}

\newglossaryentry{Mushroom}{
	name=mushroom,
	type=good,
	description={can be collected from \hyperref[ch:Goods:Nature:Plants:Limited]{plants}}
}

% Building Materials
\newglossaryentry{Timber}{
	name=timber,
	type=good,
	description={is a \hyperref[ch:Goods:Materials]{building material} harvested from \hyperref[ch:Goods:Nature:Plants:Limited]{trees}}
}

% Food
\newglossaryentry{Meat}{
	name=meat,
	type=good,
	description={gained by slaugthering \hyperref[ch:World:Inhabitants:Animals]{wild animals} and
			\hyperref[ch:World:Inhabitants:Livestock]{livestock} it can directly be consumed as \gls{Food}}
}

\newglossaryentry{Fish}{
	name=fish,
	type=good,
	description={fished from the \gls{Sea} the \hyperref[ch:Goods:Nature:Sea]{fish} can be directly consumed as \gls{Food}}
}

% Merchandise
\newglossaryentry{Jewlery}{
	name=jewlery,
	type=good,
	description={describes all \gls{Merchandise} use as personal decorative items}
}

\newglossaryentry{Clothes}{
	name=clothes,
	type=good,
	description={describes all \gls{Merchandise} use to satisfy the need for \gls{Clothing}, without any special benefits}
}

\newglossaryentry{Furniture}{
	name=Furniture,
	type=good,
	description={describes all \gls{Merchandise} used in homes for personal comfort}
}

% Equipment
\newglossaryentry{Shoe}{
	name=shoe,
	type=good,
	description={are \gls{Equipment} worn on the feet, they enhances the speed of the wearer}
}

% Tools
\newglossaryentry{Pick}{
	name=pick,
	type=good,
	description={a \gls{Tool} often used for mining}
}

\newglossary*{biome}{Biomes}

\newglossaryentry{Grassland}{
	name=meadow,
	type=biome,
	description={consists of sea of grass dotten with groves and rocks. The animals are harmless. Further information can be found in \autoref{ch:World:Biomes:Grassland}}
}

\newglossaryentry{Forest}{
	name=forest,
	type=biome,
	description={is a sea of trees, with animals roaming the thicket and the glades. Further information about can be found in \autoref{ch:World:Biomes:Forest}}
}

\newglossaryentry{Jungle}{
	name=jungle,
	type=biome,
	description={is full of buzzing insects, countless trees and dangerous predators roaming inbetween. Further information can be found in \autoref{ch:World:Biomes:Jungle}}
}

\newglossaryentry{Desert}{
	name=desert,
	type=biome,
	description={consists of dune sand rocks. It is a harsh land as rich in minerals as pure in flora. Furhter information can be found in  \autoref{ch:World:Biomes:Desert}}
}

\newglossaryentry{Mountain}{
	name=mountain,
	type=biome,
	description={made of rocky peaks rich in minerals, it is a land rich in minerals and hostile to miners. Further information can be found in \autoref{ch:World:Biomes:Mountain}}
}

\newglossaryentry{Sea}{
	name=sea,
	type=biome,
	description={are rolling waves, schools of fish and a gentle breeze. Further information can be foun in \autoref{ch:World:Biomes:Sea}}
}

\newglossary*{creature}{Creatures}

\newglossaryentry{Deer}{
	name=deer,
	type=creature,
	description={are forest dwellers. They pose no danger to other cratures and flee, when
			spotted or attacked. They are hunted for \gls{Meat} and \glspl{Hide}}
}

\newglossaryentry{Boar}{
	name=boar,
	type=creature,
	description={dwell on the grassland and the forest. They are hunted for their
			\gls{Meat} and \glspl{Hide} and often fight back against the hunters}
}


\makeglossaries{}


\begin{document}

\maketitle

\tableofcontents

\chapter{The world}

The tribes you will play in is settled by different
\hyperref[ch:Tribes]{tribes}. These dwell together with wild animals, great
beasts and precious resources in the forests, jungles, grasslands and deserts
of the world.

To protect themselves from the elements and their opponents most of them chose
to erect great fortified cities and build thriving economies, governed by their
unique approach to the problems of the world and the deities they do or do not
worship.

To lead your chosen tribe to greatness you should learn about it, the other
tribes and the world they inhabit. This document is supposed to support you to
do this.

\section{Biomes}
Let us begin by talking about the different biomes within the world. These are
more or less hospitable in general and for some tribes specifically. They are
defined by different flora, fauna and resource abundance. A short summary is
provided here.

\begin{longtable}{cc ccc}
	\toprule
	Biome     & Trees   & Fertility & Animal-Danger & Minerals \\
	\midrule
	Grassland & Low     & Medium    & Lowest        & Lower    \\
	Forest    & Higher  & Medium    & Low           & Lower    \\
	Jungle    & Highest & Highest   & Higher        & Lowest   \\
	Desert    & Lowest  & Lower     & Higher        & Medium   \\
	Mountain  & Lower   & Lowest    & Higher        & Highest  \\
	\bottomrule
\end{longtable}

\section{Grassland}
Grassland is the simplest of all biomes. It is fertile for farming, hosts
rather tame wildlife and a few tries. Between the rocks some resource veins
might be discovered.

\section{Forest}
The forest is like the grassland biome but hosts some more hostile wildlife and
an abundance of trees. Within it mushrooms and

\section{Jungle}
The jungle is lush and fertile. Rare in minerals rich in dangerous animals it
is not well suited for most tribes. If the animals can be tamed and the trees
uprooted it is the most fertile biome.

\section{Desert}
The endless dry sand and rock of the desert hosts almost no life. The animals
that roam there are dangerous. The only fertile spots are found along rivers
and oasis. While it is poor land for farming it is excellent for mining, giving
a boost to the tribes that can tame this terrain.

\section{Mountain}
Bare rock bears no fruit and so no tribe can live on the mountain alone.
Despite that and the more dangerous wildlife the rich ore veins and the
abundance of rock encourage the tribes to build mining outposts within to feed
their growing industry.

\section{Sea}
The sea is not a biome that can be inhabited by the tribes, but its bounty they
can use.


\chapter{The tribes}\label{ch:Tribes}
This chapter describes the different factions inhabiting the world. It
describes their economy, religion, culture and play-style. Each faction is
centred around a unique play-style with partly overlapping goods production,
with the other factions. For a short overview consult the
\hyperref[ch:Tribes:Overview]{glossary} or the
\hyperref[ch:Tribes:Comparisons]{comparisons}.

\section{\Gls{Vikings}}

\begin{flushright}
	\emph{One with the spirits, one with nature.}
\end{flushright}

Deep within the woods, the blooming meadows live the \gls{Vikings}. They have a
close connection to nature and its spirits, which they venerate in special
sanctuaries.

In the forest their hunters stalk deer and boars, among the druids and
foragers. The foragers collect herbs and mushrooms for the production of
medicine and (herb) flavored mead and mushroom ale. The wheat for the ale is
grown in small fields in front of their towns, where their craftsmen produce
golden-jewelry, drinking horns, pillows, woolen clothes, candles, leather shoes
and wooden and metal tools. To feed them the backers bake bread and delicious
chestnut- and apple pies. Should the miller be sick, the instead feast on meat,
fish and apples.

Their houses are built from timber, boards and stones. Their hamlets protected
by simple palisades, their villages by wooden walls and towers and their cities
by stone walls, towers and the spirits the so cherish.

While the fish are caught on the shore, sheep and oxen are raised on the
meadows and chicken and pigs fed with wheat and chestnuts. Behind the pastures,
wheat-fields and orchards you can find the mountains. If they do not turn them
into sacred hallowed places, they mine for copper, iron, gold and coal. They
use this metals to make various tool. The most beloved of this tools is the ax,
used by their lumberjacks, the butchers and the warriors alike.

If the wood-burner works diligently and enough metal is found in the mountain,
their weapon smith will forge battle-axes, spears and swords. Luckily he also
works as a bow-maker on the side supplying them with short bows.

Tho protect their warriors they manufacture medium wooden and medium metal
shields, leather and light metal armor. To protect the heads of their warriors
they use leather caps, helmets and horned helmets. The latter one are seen as
quite prestigious among them and are given to the most important warriors to
boost their morale in battle.

\subsection{Religion}
The \gls{Vikings} dedicate areas to the spirits. Within theses areas the
spirits prosper and supply the \gls{Vikings} with resources and protect these
sacred places, while they remain pristine. The sanctuaries have a inner circle
and an outer circle. One sanctuaries outer circle can not overlap with the
outer circle of another sanctuary.

All sanctuaries are employ druids. They improve the sanctuary and collect
magical resources from it. In the druid-circle great druids can be trained.
These can then use the collected magical resources to weave spells.

\subsubsection{Sanctuary of the forest}
The spirits of the forest cherish the trees and the sanctuary is strengthen by
all ancient trees in its inner and outer circle. The druids plant trees in the
inner and outer circle of the sanctuary, which produce magical resource.

\paragraph{Level 0: Sacred grove}
After the sanctuary was established the trees around it slowly turn into
ancient trees. Druids can harvest ancient bark from this trees\footnote{ Magic
	bark is just magic resource, not a special good. }. Great Druid can use the
spell \textquote{Create forest}\footnote{ A small forest appears at the target
	area of the spell. } once the sanctuary was established.

\paragraph{Level 1: Living woods}
Additional herbs and mushrooms begin to sprout in the inner and outer circle of
the sanctuary. A bear appears to defend the inner circle of the sanctuary.
Boars and deer appear in its inner circle regularly. Great druids can use the
spell \textquote{Animal kinship}\footnote{ Creates a number of boars and deer
	at the place. } once this level is reached.

\paragraph{Level 2: Ancient forest}
Ancient trees now grow chestnuts and apples. Wisps start appearing confusing
and stunning enemies walking into the inner circle of the sanctuary. The wisps
also leave gits of mushrooms and herbs at the center of the sanctuary. Great
druids can use the spell \textquote{Wisp lights}\footnote{ A few lights appear
	stunning enemy soldiers. } once this level is reached.

\paragraph{Level 3: Enchanted woods}
The sanctuary attracts a might ent. The tree-shepherd patrols the inner circle
of the sanctuary and kills enemies that dare to venture in it. He also delivers
the dead-wood of the forest as timber to its center. Thorns fill out the entire
inner circle of the forest, injuring all enemies that walk upon it. Great
druids can use the spell \textquote{The forests revenge}\footnote{ Trees within
	the area of the spell awakened. They maul close enemies and throw branches at
	them over a short distance. } once this level is reached.

\subsubsection{Sanctuary of the meadows}
The spirits of the meadows cherish all flowers on it. The druids fell trees,
remove rocks and plant flowers within the inner and outer ring of the
sanctuary.

\paragraph{Level 0: Peaceful fields}
Once the sanctuary was established the flowers begin to bloom. From these
blooming flowers the druids can harvest the enchanted petals{ Enchanted petals
		are just magic resource, not a special good. }. Great Druid can use the spell
\textquote{Spring}\footnote{ A bunch of flowers appear. Since this is the only
	way besides the sanctuary to create flowers it is a convenient way to either
	level it up faster or get mead production going. } once the sanctuary was
established.

\paragraph{Level 1: Flower fields}
A wild bull appears defending the inner circle of the sanctuary. The grass
grows lusher and sheep and oxen grow faster on the meadow. Great druids can use
the spell \textquote{The great shedding}\footnote{ All sheep in the area of the
	spell immediately produce two units of wool. } once this level is reached.

\paragraph{Level 2: Fairy circle}
A bunch of fairies fly between the flowers, in the inner circle. They can turn
enemies into sheep for a short duration. They also assist the bees collecting
honey, while shearing the sheep delivering honey and wool to the center of the
sanctuary. Great druids can use the spell \textquote{Bee hive}\footnote{ A
	swarm of bee appears and chases enemies in a random direction, while doing
	minor damage to them. } once this level is reached.

\paragraph{Level 3: Unicorn pasture}
A herd of unicorns moves into the inner circle of the sanctuary. The will
trample and pierce unwelcome intruders with their horns. They shed their horns
on occasion leaving them in the center of the sanctuary. Once the herd grows
above a certain size unicorns will join the \gls{Vikings} as mount to defend
the spirits. Great druids can use the spell \textquote{Unicorn
	stampede}\footnote{ Calls a herd of unicorns from the spirit realm, that
	trample from the position of the great druid in the direction of the spell.
	Trampling all in their way. } once this level is reached.

\subsubsection{Sanctuary of the sea}
This has to be built within a lake or the sea shore. The spirits of the sea
cherish the rough waves and the cold breeze. The druids transform adjacent land
into swamp and row out to the sea in boat to converse with dolphins and
manatees.

\paragraph{Level 0: Salty rocks}
Once the sanctuary was established the water tiles around it slowly become
rough waves. Willows grow in the swamp. From willows the and rough seas the
druids can bottle wailing{ Bottled wailing is just magic resource, not a
		special good. }. Great Druid can use the spell \textquote{Rich fishing
	grounds}\footnote{ The number of fish in the water increases, } once the
sanctuary was established.

\paragraph{Level 1: Great reef}
Countless fish now live in the reef\footnote{ Fish spawn next to the sanctuary.
} and only kept in the balance by a large shark that circles it within the
inner circle. In the swamp within the inner circle a giant toad preys on the
enemies of the \gls{Vikings}. Great Druid can use the spell \textquote{Salty
	catch}\footnote{ Used on the coast, it creates a bunch of salt and fish. } once
this level is reached.

\paragraph{Level 2: Undersea harbor}
The spirits of the sea walk on the land. Within the inner sanctuary a guard of
fish-men defends the sanctuary with their spears fashioned from (whale)-bone.
They gift the druids gold in exchange for their stories. Great Druid can use
the spell \textquote{Wet warriors}\footnote{ Used on the coast it calls a few
	fish-men, that attack nearby enemies before retreating back into the depths. }
once this level is reached.

\paragraph{Level 3: Temple of the nymph}
A nymph has moved into the sanctuary. Her beauty enchanting all that see her.
If enemies are foolish enough to attack her, she will graciously accept a few
of them as her new honor-guard. They will defend her against their former
comrades until they die of starvation. She will also gift a Mithril sword to
the \gls{Vikings} from time to time to thank them for the protection of the
sanctuary. Great Druid can use the spell \textquote{Sirens song}\footnote{ Used
	on the coast it calls all enemies towards the shore, where they strip
	themselves of weapons and armor and start to swim towards the siren until they
	drown. } once this level is reached.

\subsubsection{Mountain sanctuary}
This has to be built next to mountain and rocks. The spirits appreciate rock
carvings and moss. The druids carve runes on rocks or lift rocks from the
ground\footnote{ Since rocks yields stone this should be a very slow process.
}.

\paragraph{Level 0: Rock field}
Once the sanctuary was established crystals start to sprout on the carved
rocks. The light of this crystals can be harvested by the druids{ This yields
		magic resource. }. Great Druid can use the spell \textquote{The bones of the
	earth}\footnote{ A bunch of rocks appear in the target area.footnote{ These can
			be harvested for stone. } } once the sanctuary was established.

\paragraph{Level 1: Crystal grotto}
A lonely cougar moves into the inner circle of the sanctuary. It attacks all
enemies of the \gls{Vikings}. Carved-Rocks covered in crystals, explode after a
time, yielding a small amount of stone. Great Druid can use the spell
\textquote{Riches of the mountain king}\footnote{ The resources of mines in the
	area are increased by a small amount. } once this level is reached.

\paragraph{Level 2: Troll cave}
A mighty troll moves in. With his mighty pick the troll not only keeps the
inner circle save of invaders, but also mines the occasional piece of copper
ore from the crystals depositing it its hoard in the center of the sanctuary.
Great Druid can use the spell \textquote{You shall be stone}\footnote{ Turns a
	small amount of enemies and their weapons permanently into stone/rocks. So the
	houses of the \gls{Vikings} can be built from their enemies. } once this level
is reached.

\paragraph{Level 3: Dragon peak}
A mighty dragon takes the center of the sanctuary as its domain. It defends the
inner sanctuary with its flaming breath from invaders and permits the druids to
collect its iron scales once they shed. Great Druid can use the spell
\textquote{Dragons curse}\footnote{ Turns a small amount of enemies and their
	weapons permanently into copper-, gold- and iron-bars. } once this level is
reached.

\subsubsection{Sanctuary of the hearth fire}
The spirits of the hearth fires love the \gls{Vikings}, their fields and
houses\footnote{ The sanctuary levels with wheat fields and buildings in its
	inner and outer circle. Making it different other sanctuaries. }. The druids
walk from house to house lighting candles\footnote{ So this is the only
	sanctuary that consumes resources. } to delight the spirits.

\paragraph{Level 0: Hamlet}
Once the sanctuary was established the burnt candles turn into spirit-embers,
which can be collected by the druids\footnote{ Spirit embers are magical
	resource not a special good. }. Great Druid can use the spell
\textquote{Wonderful harvest}\footnote{ A bunch of ready to harvest wheat
	appears. } once the sanctuary was established.

\paragraph{Level 1: Village}
A group of mocking spirits appear and wander the inner circle of the sanctuary.
Their mocking distracts enemies stopping them from attacking. All shops and
fields passed by the spirits gain a production boost. If the spirits pass an
empty pigsty or chicken-pen a pig or chicken appears in it. Great Druid can use
the spell \textquote{Spirits gifts}\footnote{ A bunch of clothes and bread
	appear. } once this level is reached.

\paragraph{Level 2: Town}
A ghostly night-watch appears and defends the inner circle against intruders at
night.\footnote{ The night-watch are ghostly soldiers armed with spears. }. At
dawn the night watch turns into alcoholic beverages. During the day a burning
knight walks through the inner circle of the sanctuary. Ready to defend it
against invaders. At dusk the knight turns into a few units of coal. Great
Druid can use the spell \textquote{Black rider}\footnote{ A black ghostly rider
	appears. Slaying one enemy, causing nearby enemies to flee in panic. } once
this level is reached.

\paragraph{Level 3: City}
A group of animated armor suites patrols the inner circle of the
sanctuary\footnote{ They wear metal armor, helmets and either bows or swords
	and shields. }. From time to time the armors are remade by the sanctuary
leaving their components behind for the warriors of the \gls{Vikings}. Great
Druid can use the spell \textquote{The great bell}\footnote{ A giant bell tolls
	three times shattering walls beneath it. } once this level is reached.

\subsection{Starting conditions}
The \gls{Vikings} start with a great wooden hall. This is the center of their
community, where they can store resources and train warriors. The balconies of
the hall can also be manned by bowmen and the doors locked Turning this
building in a defensive point.

\subsection{Play-style}
They are rather generic, growing food, mining for ore, felling trees. Their
gimmick are the sanctuaries, forcing them to carefully consider the use of
their territory. The sanctuaries give them a great natural defense and infinite
resources for little work. On the offensive they can rely on their medium
warriors, but are better of to use the spells of their great druids. In the
early game they can use their tool-smithy to forge axes and start raiding their
neighbors, before those build a working weapon production line. For
siege-warfare they rely on ladders and \textquote{The great bell}-spell. With
their ability to build longboats and upgrade their buildings from simple tents,
to huts, to wooden houses and then to stone houses, they are quite static. They
usually defend themselves with palisades, wooden walls and towers and in the
late game with stone walls.

Their true weakness is the desert, because they have neither sanctuary nor use
for it. They can also only farm on rare fertile soil and their yields are small
compared to other people.


\section{\Gls{Nomads}}

\begin{flushright}
	\emph{The mount below, the wind behind, ahead the enemy.}
\end{flushright}

Among the cruel sun of the endless steps live the \gls{Nomads} in their tepees
and yurts. All expect their holy places can be move\footnote{ This also means
	that the naturally spawned defenders of these holy places are more powerful
	than their \gls{Vikings} equivalents. }. They live in tepees, yurts and wagons.
The foragers, cattle-breeders, artisans\footnote{ The artisans make weapons and
	tools from wood and leather. }, mount-breeders, butchers, lumberjacks,
meat-driers, warriors, shamans, cloth-makers and chieftains can move their
homes and workshops with ease. Since tools, tepees, yurts and wagons are made
from leather timber and woolen-cloth the economy of the \gls{Nomads} is quite
small.

\Gls{Nomads} live of their large herds. In the sunny desert they bread camels as mount
and cattle, in the temperate plain they bread oxen and horses and in the frosty north
they bread mammoths and yaks.
They survive of the dried meat of their cattle and drink their fermented milk for
entertainment.

Their workers wear woolen and leather clothes and boots, their warriors leather
armor and helmets. They produce neither shields nor metal blades. Therefore
their warriors carry spears, lances and short bows into battle.

The only luxury goods they produce is furniture and great coats. They are a
simple people and fight for the three gods of the plains, the endless sky above
their heads, the endless ground below their galloping mounts and the sun
shining on guilt and non guilty alike.

\subsection{Religion}
\Gls{Nomads} have three gods that all require different offerings.
For this offerings the shamans can cast spells, two per temple level.
One is always a boon the other a curse.
Once a temple has enough offerings it levels up.

\subsubsection{Temple of the sky}
The god of the sky demands mounts as offerings. The wind is mighty, fickle and
destructive. It bestows its power upon those that can ride with it, enabling
them to lay their foe low.

\paragraph{Level 1: Sky altar}
A swarm of crows takes the offered mounts up into the skies. Multiple swarms of
crows defend the temple attacking all enemies that come close doing small to
medium damage. The boon of this level is \textquote{Wind in our
	backs}\footnote{ Mounted warriors ride faster for a while after being affected
	by this spell. }, its curse is \textquote{Murder of crows}\footnote{ A swarm of
	crows appears and can be controlled by the player for a time. While they can
	damage enemies they might be better suited as scouts. }.

\paragraph{Level 2: Harpy peak}
A swarm of harpies appears and harasses all enemies that come close to the
temple. They do not only damage them but also sometimes stun them with their
shrieks. The mounts are now collected by a giant bird. The boon of this level
is \textquote{Long arrows}\footnote{ Affected bowmen get a significant range
	increase. }, its curse is \textquote{Winds of change}\footnote{ A strong wind
	starts to blow destroying trees and crops below it. }.

\paragraph{Level 3: Sanctum of the winged sovereign}
A giant bird appears on the temple to feasts on the mounts. It flies off to
attack enemies in the direct vicinity of the temple.

The boon of this level is \textquote{Great hunt}\footnote{ Mounted warriors
	affected by this spell, start to fly for a short while, permitting them to
	cross seas, mountains and city walls. They land once the stand still and there
	is a place to land nearby. If the spell wears off and there is no landing place
	they keep flying. }, its curse is \textquote{Great talons}\footnote{ A giant
	bird appears and destroys whatever is under it, be it trees, city walls or
	small enemy armies. }.

\subsubsection{Temple of the earth}
The god of earth is simple as the dirt below you feet. In exchange for meet it
grants you the power of the beast. The earth is steady and fair, it will feed
your people and see to their protection.

\paragraph{Level 1: Altar of bull}
A group of wild bulls patrols the area around the temple attacking all enemies.
Wild cattle spawn next to it. The boon of this level is \textquote{Gifts of
	mother earth}\footnote{ People affected by this spell will loose hunger, }, its
curse is \textquote{Tremor}\footnote{ Affected people will be stunned by the
	shacking earth. Buildings will be lightly damaged. }.

\paragraph{Level 2: Shrine of the Minotaur}
A Minotaur guard appears to defend the temple. Wild mounts spawn next to the
temple. The boon of this level is \textquote{Fertility}\footnote{ A few of the
	affected cattle and mounts immediately procreate. }, its curse is
\textquote{Curse of the horns}\footnote{ A group of enemies is temporary turned
	into cattle, leaving their equipment on the ground. This means once the spell
	ends they enemies still disarmed. }.

\paragraph{Level 3: Hall of the horned king}
A Minotaur chieftain moves into the temple defending it and occasionally eating
close by cattle. The boon of this level is \textquote{Blessing of the
	horns}\footnote{ A group of people sheds all their armor and humanity. They are
	turned into Minotaur, giving them twice the strength and appetite for meat. The
	lack of protection they compensate with their speed and the additional power of
	their blows against the enemies. }, its curse is \textquote{Thunder on the
	plains}\footnote{ A bunch of ethereal cattle appears and trampling and injuring
	everyone in their path. }.

\subsubsection{Temple of the sun}
The sun is as benevolent as vengeful. It is prideful and demands only complete
devotion from its followers. Remember it is foolish to oppose the will of the
sun\footnote{ Only one sun temple can be built, to avoid mass worship. }.

\paragraph{Level 1: Sanctum of the sun}
The sanctuary can be worshiped by the \gls{Nomads}. This will let them carry
favor with the sun. Around the temple sun-rays blind enemies, stunning them
shortly. The boon of this level is \textquote{Healing ray}\footnote{ The sun is
	benevolent. Once it touches the wounds of your warriors they will magically
	heal. Praise the sun. }, its curse is \textquote{Exhausting heat}\footnote{ The
	sun burns down on your enemies, drenching all the sweat from their body. This
	will slow them and their mounts down significantly. }.

\paragraph{Level 2: Council of the serpents}
Some of the worshiping nomads will turn into snake-humanoids. These will defend
the temple with their cooper swords, spears and shields. Once a certain number
of snake-humanoids is reached all further converted worshipers will join the
armed forces of the \gls{Nomads} instead. The boon of this level is
\textquote{Feathered mounts}\footnote{ The sun sends down mounts from a
	forgotten time. Those giant birds are not only faster than your mounts they can
	also bite the enemies had clear off. }, its curse is \textquote{Curse of the
	scale}\footnote{ The sun shines on all equally, so these cure affects friend
	and foe alike. A few warriors hit by its rays drop their armor and weapons,
	before turning into beasts from a forgotten time. These beast start attacking
	the closest people they find. }.

\paragraph{Level 3: Temple of the blinding empress}
A T-Rex appears and patrols around the temple. It attacks all enemies of the
\gls{Nomads} that dare to venture close. The boon of this level is
\textquote{Great beasts of burden}\footnote{ The sun sends down great beasts of
	burden from a forgotten time. These can carry your tepees/yurts/wagons faster
	than normal mounts. They can also carry saddles, carrying you craftsmen
	workshops, permitting them to work while on the move. }, its curse is
\textquote{Will of the sun}\footnote{ The sun shows its face and basks the area
	in its glory. Every living creature in the center of the spell is immolated.
	All burnable things catch fire. All living things around the center are
	injured, disoriented and are severely injured. }.

\subsection{Starting conditions}
\Gls{Nomads} start with a chieftain-tent, a mobile scouting/defense tower\footnote{
	A platform for a few bowmen that can be upgraded to give more visibility
	and accompany more bowmen.
	This is their only defensive building and it is weak against ground attack.
	To weak to be a really viable defense.
	But still better than being butchered on the ground.
}, a few living tents, a cattle-breeder camp and a few mounts and cattle.

\subsection{Play-style}
\Gls{Nomads} are a horde faction in more way then one.
They have no defenses so their best defense is a good offense or a fast retreat.
Considering that they can speed up the process with which they can move
their base both is an option.

Their true strength are their mounted archers allowing them to harass their
enemies as they please. The lack of a complex economy permits them to focus
fully on plundering the resources they need. \Gls{Nomads} should be always on
the attack while growing their herds behind the lines, relocating their camp to
avoid enemy retaliation. This way they will weaken others while gaining
strength. Their true powers lies with their shamans and the sacrifice they can
make to their gods.

The ability to upgrade their tepees to yurts and then to wagons improves
productivity and mobility, while increasing the number of mounts necessary for
transport\footnote{ A tepee can be loaded onto a mount or carried by a human. A
	yurt has to be loaded onto a mount. A wagon needs to mounts. }.


\section{\Glsfmttext{Egyptians}}\label{ch:Tribes:Egyptians}

\begin{flushright}
	\emph{Behind high walls, build for immortal glory.}
\end{flushright}

Along the fertile plains of the river irrigated fields of reed are tended and
glorious monuments erected. Within their palaces rule the priest-kings of the
\gls{Egyptians} over their people, protecting them in this life and the next.

The \gls{Egyptians} build great cities from nothing stone. Forgoing wood,
bricks and other materials their buildings store the cold of the night during
the scorching days in the desert, while being immune to fire. The only
buildings that they need to maintain are their monuments, which they furbish
with rich fabrics and paint.

Should the maintenance of the monuments be neglected, their glory will diminish
and the \gls{Egyptians} will rise against their priest-king to install one that
ensure their glory lasts eternally.

Their agriculture is based on the cultivation of wheat, flax and papyrus. They
raise small cedar forests to collect timber and resin. The wheat is milled into
flour and baked into bread, which they enjoy with fish, goat cheese and milk
and beer. As a additional treat they like to snack on figs, which they raise in
special orchards.

\Gls{Egyptians} were strongly influenced by the stark contrast between
the river and the desert in their homeland,
leading to an agriculture that is strongly reliant on irrigation.
While they may farm of simple fertile soil their yields can be drastically
increased by irrigation.
To facilitate the irrigation they build \textquote{Shrines} next to a body of sweet water.
Within the shrine resides a singular priest and their assistant.
The assistant uses a shovel to dig the ditches and maintain them.
The priest supervises the work and prays for the blessing of the local river.

Besides their local river the \gls{Egyptians} pray only two three other gods:

\begin{enumerate}
	\item The god of family and hearth fire, which they worship within their houses in
	      private. If they can obtain them by trade, they born candles in his honor. He
	      sometime reciprocates by small gifts of gold.
	\item The goddess of war. She is worshiped on the battlefield and her temple also
	      serves as mustering ground for new potential warriors. The priests bless all
	      unarmored soldiers, so they fight with greater zeal.
	\item The king of the afterlife, who was slain, cut to pieces and strewn across the
	      world. His priests are the priests of the mortuary cult. They work tirelessly
	      to prepare for his resurrection. To this end they practice the preservation and
	      reanimation of the dead. To ensure not a single soul is lost, they equip all
	      soldiers of the \gls{Egyptians} with magical talismans that teleport their
	      armor, weapons, shields and corpses back to the closest mortuary temple. There
	      they are stored until they can be wrapped in linen and anointed with resin,
	      before they are stored in a necropolis nearby.
\end{enumerate}

Their craftsmen weave flax into linen and sow linen into linen clothes. The
clothes are worn for protection against the environment and can be colored to
increase their value and the satisfaction of the wearer. The paint for this
fine linen clothes is made from resin and figs. The papyrus is crafted into
paper, on which a poet using the aforementioned paint writes beautiful poems.
These are traditionally burned after being read to ensure the experience
remains unique. These public readings in the theater are attended by many of
the \gls{Egyptians} and satisfy their desire for beauty. The only other way are
masterful wall-paintings, but these loose their appeal after a time, so the
painter needs to paint a new scene in the house after a while.

The carpenter cuts down the large cedar logs and turns them into furniture.
Since the houses of the \gls{Egyptians} are no place for heavy wooden
furniture, the carpenter uses raisin and linen to build light elegant cots and
chairs. The leftover timber is the usually sold on small river-galleys built
from it. The sails of this galleys are obviously made from timber.

To build stronger war galleys a ram is added and often reinforce with a metal
tip. The \gls{Egyptians} only mine gold, copper and coal themselves. From the
gold the goldsmith forges dead-masks and golden-jewelry. The copper is forged
into tools and swords and spears. They also use short bows and medium wooden
shields. Their hats are protected by either linen caps or copper helmets. Their
warriors also wear linen armor\footnote{ This is equivalent to leather armor. }
or light metal armor. They also breed mounts to carry the chariots, their
chariot maker furnishes from copper, wood and linen. These chariots need to be
manned by an unarmed driver and can carry up to two soldiers into battle.

In general \gls{Egyptians} prefer to build and farm instead of fighting. So
they raise impressive walls and towers from stone. While they might be rather
small in the beginning they keep rising\footnote{ They can upgrade their walls
	continually, without making them unusable. } until they are only rivaled by the
\gls{Romans}. From there their archers rain death upon their enemies until the
army of \gls{Egyptians} is strong enough to beat their enemies on an open
field.

\subsection{Monuments}
The monuments are testament to the right to rule of the local priest-king or
-queen. Their glory is the legitimacy of the current ruler. To ensure they
remain glorious monuments instead of becoming abandoned ruins they have to be
constantly refurbished. The resources necessary for this are usually linen,
furniture and paint\footnote{ If not otherwise sated all resources have to be
	supplied to maintain the glory. So having a ton of linen and no furniture may
	lead to a glory issue. }. Should the glory be insufficient for the number of
subjects currently ruled they will riot.

\subsubsection{Painted wall}
This is a wall that is regularly painted with new scenes glorifying the current
monarch. Since the stone wall was built to last for aeons, only paint is
necessary to maintain this monument. Calling it a monument is kind of a stretch
however, motivational billboard might be more fitting.

\subsubsection{Public house}
A simple luxury. A house filled with fine furniture and splendid wall
paintings\footnote{ It is not necessary to resupply color and furniture. With a
	full supply of either half the glory can be achieved. For the full amount of
	glory both are necessary. }. The constant change of furniture and scenes shows
the prosperity of this small realm and invites the subjects to relax.

\subsubsection{Clothed statue}
This statue is painted and clothed in linen. It is a over-sized representation
of the local ruler. To show the glory of the ruler the face paint and clothes
have to be always following the newest style, so they are constantly changed.

\subsubsection{Festival Plaza}
This is a splendid building decorated with gold and exquisite furniture. Within
it \gls{Egyptians} celebrate the glorious rule of their current monarch. To do
this in the appropriate fashion they require food, dates and copious amounts of
beer\footnote{ The only necessary resource is beer. All other can be supplied
	independently. Every resource increases the glory by 20\% supplying all gives
	another 20\%. }. Since the celebrations can become quite rowdy the furniture
also needs replacement from time to time.

\subsubsection{Great veiled theater}
This place is one of exotic entertainment. Every play is held behind a
transparent current of linen illuminated with coals from behind. All the actors
are only visible as shadows and the play is entirely silent. During the play
the spectators try to guess the identity of the actors and which role they
play.

After the play has finished the plot is read out aloud to satisfy the curiosity
of the spectators. Afterwards the linen curtain and the pages the play was
written on are burnt to ensure the next play will hold a new mystery.

\subsubsection{Palace}
This is the seat of a priestly monarch. To maintain the glory of this palace
its walls needs to be repainted, the linen curtains replaced and the furniture
updated to the newest tastes\footnote{ Every resource (paint, linen, furniture)
	gives 25\% of the total glory value. Supplying all 3 gives another 25\% }.

The palace differs from other monuments, because it also doubles as a defensive
structure. Its inner compound is flanked by small towers that allow archers to
defend it. This makes it an interesting choice in a forward position.

\subsubsection{Great Library}
The great library contains countless works of art and long forgotten knowledge.
On fine chairs the scholars lounge and amuse themselves with the newest
creations of the poets. Curiously the best poems always seem to go missing, so
that a constant stream of replacements has to be provided\footnote{ The
	building consumes only papyrus. }.

\subsubsection{River Baths}
The bath is built on the shores of the river. Coal is used to heat the sauna
and the pools and fresh linen-towels provided to every guest. A luxury that
shows the glory of the \gls{Egyptians}.

\subsubsection{Halls of contemplation}
This hall is filled with elegant furniture stone statues and countless
meditation candles\footnote{ Consumes furniture and candles. This implies the
	\gls{Egyptians} are trading with the \gls{Vikings}. }. The pristine calm that
fills its visitor is a feeling that no one else achieves. Together with the
sublime architecture on the inside the splendor is truly glorious.

\subsubsection{Relaxation chambers}
On the divas woolen cloth and soft pillows invite the visitor to contemplate
the sublime paintings on the wall\footnote{ The building consumes paint,
	pillows and woolen cloth. This implies the \gls{Egyptians} are trading with the
	\gls{Vikings} and the \gls{Nomads}. }. A tribute to the far reaching trade
connections of the \gls{Egyptians}.

\subsubsection{Hall of foreign delights}
In these halls food and drink from all over the world are presented\footnote{
	All non-domestic foods and drinks contribute. There is a small bonus for every
	additional, that slowly grows up to 25\% of the total glory. }. Demonstrating
the prosperous trade-routes of the \gls{Egyptians}.

\subsubsection{Intercultural exhibition}
Goods from all over the world are exhibited here\footnote{ All non-domestic
	foods and drinks contribute. There is a small bonus for every additional, that
	slowly grows up to 25\% of the total glory. }. Unfortunately they keep
disappearing, so replacements are necessary all the time.

\subsubsection{Tomb}
This is the tomb of a member of the royal court. The owner needs to be needs to
be comfortable by changing the outer layer of linen around him every day.

\subsubsection{Great Tomb}
In this tomb lays an important member of the royal court. To ensure his comfort
in the after live the outer layer of his linen wrapping is changed daily and
new furniture is burnt in front of a small altar.

\subsubsection{Royal Tomb}
A member of the royal family is buried here. In addition to the usual linen
wrappings and furniture offerings they were a number of golden masks. These
have to be reforged from time to time\footnote{ The monument turns golden mask
	into jewelry. }.

\subsubsection{Pyramid}
The pyramid is the greatest monument. Is size alone is glorious\footnote{ A
	quarter of the glory stays even if there is no maintenance. }. Within the
pyramid lies an important member of the royal family. So in addition to linen,
furniture, and golden masks the family insists of offer fine linen cloth,
dates, bread and beer in equal measures, to ensure the deceased has a decent
existence in the afterlife.

\subsection{Religion}
As can be quickly seen the mortuary cult is the most venerated and biggest of
the cults. It priest do not only embalm the dead they also maintain them within
the large necropolis, that the \gls{Egyptians} raise should they be engaged in
war. Within those complexes the mummies of the warriors lay next to their
weapons. From time to time the priests replace the outer layer of linen to
ensure the dead warrior is comfortable in the after-life. It is by this action
that the mortuary cult gains favor with the spirits of the afterlife.

These spirits grant them power over the afterlife itself. So the great priests
of the mortuary cult can exchange this favour to cast the following spells:

\subsubsection{Least death}
The most meager servant of the afterlife appears and takes the life out of all
plants in an area. Crops disappear and trees turn into coal.

\subsubsection{Lesser Death}
Death is owned a debt an it is collected int he lives of all farm animals in an
area. They wither away and become dust. Punishing their owners for opposing the
\gls{Egyptians}.

\subsubsection{Great Death}
The doors to the afterlife open and swallow people next to them. Their bodies
wither and decay as their souls are sucked into the empty plains of the former
realm of the \gls{Egyptians}s greatest god. Only their equipment, weapons,
tools, armor, clothes remain. A strong warning for all that wish to oppose
death.

\subsubsection{From aeons past}
The lands have seen countless battles and the spirit lingers. With this spell
souls are called from the afterlife and their bones reform from the ground.
With their bone-spears and shields they fight for the \gls{Egyptians} before
they and their equipment return to the dust they were made from.

\subsubsection{You shall rise}
Enemies fallen in the affected area are facing a stark choice. Fighting for the
\gls{Egyptians} for a few moments after their death or waiting eternity at the
gates of the afterlife. For most it is a easy choice.

\subsubsection{Your duty is not yet done}
Instead of teleporting back to a necropolis the affected soldiers resurrect
directly on the battlefield. The process drains most life from their bodies. so
they mummify immediately. Beware however that the spark of live that would be
necessary to return their bodies back to the temples of the mortuary cult has
left them. So these warriors will fall and lay were they are slain again. Food
for carrions birds. Desperate times require desperate measures.

\subsubsection{His weeping widow}
The king of the afterlife left behind his bereaved wive. As the gates to her
palace are torn open, her tears fall on the land infesting it with a deep
sadness. Nothing wants to live were the tears have fallen and the area turns
into desert.

\subsubsection{His vengeful son}
The king of the afterlife has a prideful son, who wishes to avenge his father.
Driven to the brink of madness by his grief he kills all that are close.
Opening the gates to his palace allows him to sally forth in his chariot and
slay all that oppose him.

\subsubsection{His singing daughters}
After their father died, their mother became inconsolable by grief ad their
brother a slave of revenge, the daughters of the king of the afterlife began to
sing hie eulogy. Opening a door to their palaces permits mortals to hear their
song. All that hear it are overcome by a great stunning grief and are unable to
move or act as they contemplate the loss of a benign god protecting their souls
after death.

\subsubsection{Vision of the past}
A door opens showing the glory of days long gone. Workers in the area work
harder.

\subsubsection{Vision of the future}
Reveals the emptiness of the afterlife to all mortals in the area. This causes
them to despair throw away their weapons and flee.

\subsubsection{Raise oh King}
Used on a tomb, great tomb or pyramid this spell creates one or multiple undead
chariots. The tomb is destroyed in the process

\subsubsection{One more time, march to glory}
Used on a necropolis this spell raises all the warriors resting within. They
march out of its gates with the weapons they were buried. This can either be
desperate last stand or a final glorious victory, because the spark of live
that would be necessary to return their bodies back to the temples of the
mortuary cult have long left them. So these warriors will fall and lay were
they are slain again. Food for carrions birds as they were always meant to be.

\subsection{Starting conditions}
The \gls{Egyptians} always settle close to a source of fresh water, where they
erect a palace for their first priest-king or -queen. The palace itself is
surrounded by a wall and small towers forming a first defensive compound. In
front of it are a few houses for the kings retainers and servants.

\subsection{Play-style}
The \gls{Egyptians} are strong in the defense and can grow very strong in a
small area. So they usually erect a citadel along a river valley and keep
increasing their populations size and economic prowess until they can devastate
their enemy in great human wave tactics.

They only reason to leave the safety of the fortress is trade for foreign
luxury goods or the hunger for stone. This hunger for stone is driven by the
need for ever greater monuments. The true limit for the number of the
\gls{Egyptians} a single priest-king can rule. This either necessitates mining
outposts, trade or expeditions, to obtain the stone and metals the
\gls{Egyptians} lack in their citadels.

The mortuary cult also encourages early aggression to permit access to the
spells of the \gls{Egyptians}. The player has to balance this and a loss of
defense. Usually a bunch of lightly armored archers will be ordered to perform
a suicidal attack to fill the necropolis as soon as the weapon production
starts rolling, but these archers might be missing on the walls when the enemy
fights back.

If the \gls{Egyptians} is permitted to raise their monuments undisturbed they
will grow in strength until they finally reopen their necropolis. Than the
living and the dead march together for the greater glory of their kingdom
crushing everything in their path.


\section{\Gls{Aztecs}}

\begin{flushright}
	\emph{The sun hungers, our blood sustains it.}
\end{flushright}

In the deep jungles the \gls{Aztecs} erect their cities. The pyramids built
from giant stone blocks tower over wooden huts. In front of the palaces and
forts of the priestly aristocracy the massed of sweat and labor to feed the
priests and heir gods.

The \gls{Aztecs} grow beans, maize and cotton in chinampas on the lakes or
fertile soil within the plain. While the crops on the chinampas soak up
moisture over time, the fertile lands needs to irrigated by rain, which only
falls occasionally or by godly intervention. While the beans will grow quite
fast, cotton and maize will only grow if properly fertilized. While charcoal
made by the charcoal maker can fulfill the need for fertilization, the
\gls{Aztecs} have more efficient methods available if they keep the serpent
happy.

Their foresters plant the small saplings that will slowly grow to giant jungle
trees. Between this trees avocados, pineapples, tomatoes and pepper are grown
in special spots. At the center of this spots monuments are erected to speed
the growth. These are collected within small huts built into the giant jungle
trees. For timber their foresters saw giant branches of the trees or cut down
all the trees foreign to them. As such the \gls{Aztecs} life in harmony with
the forests surrounding their cities.

The giant trees within the forest are covered in veins. To ensure easy passage
through the forest special vein-cutters are employed. Those cut the veins with
obsidian swords (Macuahuitl) along predetermined paths. Only them, the
collectors and the foresters posses the skill to travel through the veins
unimpeded.

Once the veins are cut and the treasures of the forests are collected they are
brought towards the cities craftsmen. There within small market stalls the
delicacies are prepared. While the \gls{Aztecs} will eat all food if famine
looms, under normal circumstances they have quite particular tastes. The
simplest stands just serf a paste of ground up beans. Their taste is bland and
the bloating not becoming of the higher orders of their society, so only the
lowest peasants will accept this food, while the slaves are left no other
choice. The craftsmen will expect the paste to be enriched with either avocado
or pepper, but they prefer a stew made from maize, pepper and tomatoes. This
food will satisfy the peasantry or serve as an emergency rations for the
warriors. The warriors subside a spread of avocado and tomato served on flat
bread made from maize, which the craftsmen consider acceptable and the peasants
luxury.

To feed the priests that rule the \gls{Aztecs}, ponds are dug into the earth,
within these ponds eels are fattened with maize taken from the fields. These
eels are then either transported to the dog breeder or served with avocado
paste, pepper and tomatoes as special treat to the craftsmen, the warriors
common diet or the poorest of the priests. A luxury for the warriors and the
staple of the priests are eels with a pepper-pineapple sauce and avocado-paste
filled flat bread on the side. To truly satisfy the priests the tiny dogs are
butchered filled with slices of pineapple, eel, avocados and served on a bed of
tomatoes with a little pepper. If the dogs are not eaten they are carried
around by the priests of the \gls{Aztecs} as status-symbols.

As different the \gls{Aztecs} are in their taste for food, they are quite
united in their taste for drink. The simplest plainest drink they say is the
ale they brew from maize. The drink for very day is pineapple-wine. The
greatest luxury is spirit distilled form maize-ale, flavored with
pineapple-wine and a small infusion of pepper. But due to its rarity the spirit
is only rarely consumed.

However the craftsmen of the \gls{Aztecs} produce more than just food and
drink. From timber lumber is sawn, which is used to build the houses of the
common people or the furniture within. What is not turned into housing or
furniture is used to fire the potters kiln, where the clay from the clay-pit is
formed into beautiful ceramics which always seem to break.

While the stone-cutters primarily work to supply the ever growing pyramids and
palaces, the stones are also used by tool- and weapon-makers of the
\gls{Aztecs}. From broken rock and timber they furnish hows for the farmers,
axes for the foresters, and knives for the dog breeders.

For clothing they weave cotton into simple garments for the peasants and
craftsmen, woven armor for the warriors and robes for the priests.

The woven armor protects the \gls{Aztecs} just like a leather armor would. For
weapons they use slings made from cotton, atlatls made from timber and cotton
or Macuahuitl made from the sacred obsidian. If they wish to take prisoners
they use heavy wooden clubs. A single of their warriors might not seem to stand
much of a change against the other people, but the \gls{Aztecs} never fight
alone.

To protect their large number the \gls{Aztecs} erect large earthen ramparts,
which sides are reinforced by stone. The reason the \gls{Aztecs} have no direct
siege weapons, might be that only the mightiest of siege engines stand a chance
to grind them down, so the siege tactics of the \gls{Aztecs} focus on
overwhelming numbers. They prefer to pepper their enemies with stones from
their slings, before surging ladders to the walls and engage them in bloody
melee. For others this may seem absurd but it permits the \gls{Aztecs} to bring
their large numbers into play and take as many of the enemies prisoner as
possible.

\subsection{Religion}
Central to the \gls{Aztecs} religion is the ever-dying sun. At dusk the sun god
enters the under-world here he battles against the demons of darkness to save
the world. As the sun is mighty it defeats the demons with ease but their
attacks leave wounds. From this wounds the suns life-force oozes out as light.

So every dawn the sun rises to recover its lost vitality, while its life force
warms the surface of the world. But to mend its wounds the sun requires
sustenance, this sustenance can only be provided by the great serpents, which
it consumes once the reach maturity.

The legends of the \gls{Aztecs} say that once the sun is forgotten, the temples
abandoned and the great serpents day out, the sun will shine brighter every day
until all the world burns in its light. And after the surface evaporates the
upper and the underworld will merge and the sun will finally die, before a dark
sun is borne from the sprawling masses of the demons of darkness.

The \gls{Aztecs} have many gods, but for them only the sun rules supreme. All
other gods are just seen as helpers or tools to ensure that the sun does live
eternal. They see themselves as the only bastion against the rising of the dark
sun. Their temple show their reference for the gods they honour, but foremost
the temples are part of their always expanding efforts to mend the sun, and
bring about the age of blessed twilight, were the sun is healed so far that
after its rise it darkens as its mends with the serpents offered by the
\gls{Aztecs}.

\subsubsection{Shrine to the crying god}
Among the gods the crying god is most fond of humans and the \gls{Aztecs} in
general. He abhors their death, and whenever a human dies hie tears fall down
from the sky soaking the grounds.

It is said that ages ago he was the laughing god, a god of joy and laughter,
ruling supremer as god of the simple farmers, that were the ancestors of the
\gls{Aztecs}. But after the \gls{Aztecs} were called to nurture the ever dying
sun, he was pushed from his place and now watches in sorrow at the daily
atrocities they commit to save the world.

While he can not save the world he is still dear to most of the \gls{Aztecs}.
At his shrines young lovers pray for happiness, parents for the return of lost
children, priests for forgiveness and warriors for a painless death. All this
prayers have to be uttered in secret though, because there is only one ceremony
officially permitted. The sacrifice for rain. Up on the altar of the shrine a
human is killed in front of the crying gods eyes, so his tears flow from the
sky and water the fertile fields around, so the crops can grow.

\subsubsection{Temple cooking gods}
Food plays a central role in everyday life of the \gls{Aztecs} and its
religious aspects are governed by a twin deity. The cooking gods.

It is said that as the world was young the cooking gods tried to find
appreciation for their meals, but no spirit nor god ever liked their maize
gruel. So they decided to shape people from the gruel to and feed it to them.
If they praised the food the cooking gods sent them out into the world to
praise the height of their craft. If the people however did negatively critique
their cuisine they returned them to the gruel they were made from. The last
part of the story is mor often told to picky eaters than the first one.

Apart from smaller ceremonies focused on food and kitchen courses the large pot
is used for the great gruel. The great gruel is prepared by the priests from
water an maize. Once it starts boiling a member of the \gls{Aztecs} or a
prisoner can be drained of most of their blood. One the blood mixes with the
gruel the priests speak their prayers and from the pot a member of the
\gls{Aztecs} arises, as pale as the one drained earlier. Both wander off into
the city to take a few hearty meals, once their blood has slowly replenished
their are again just normal member of the \gls{Aztecs}. In this way the
population of a city governed by the \gls{Aztecs} can swell faster than that of
any other people, if enough food is available.

\subsubsection{Temple of the burning mountain}
The god of the burning mountain dwells in the volcano next to the great capital
city. There on its slopes the \gls{Aztecs} first encountered pepper, which for
them resembles the life-force the burning mountain absorbed from the sun. They
harvest these and sacrifice it to them, so that is fiery breath melts stone
into sacred obsidian.

This temple is hard to construct but necessary for every great city of the
\gls{Aztecs}. Without it the veins of the jungle impede their everyday
movements and their soldiers have to fight without their feared Macuahuitl.

\subsubsection{The serpent stable}
At the center of every settlements the \gls{Aztecs} found is a holy sky
serpent. This is not only often true in a geographical sense, but more so in an
economical, religious and cultural sense. All activities the \gls{Aztecs}
perform are aiming to fatten this serpent, so it can be consumed by the
ever-dying sun. Should they neglect this duty for to long the serpent may
escape its prison and feast on the \gls{Aztecs} in its proximity.

Should enemies approach the serpent will leave its pit and feast on them. It
teeth are sharp and bite even through the hardest metal, so the only limit to
the devastation it causes in the enemy ranks is the capacity of its stomach and
the speed of its teeth. In its juvenile form only a well equipped band of
raiders with either great strength in numbers or excellent ranged weaponry has
a chance of piercing its scales. So even without a single warrior the
settlements of the \gls{Aztecs} are well defended.

\paragraph{Pit}
Every settlement of the \gls{Aztecs} is founded by digging a pit. Within the
pit a single sky serpent egg is placed. Around it a wooden stockade is erected
to protect the egg from wild animals and scoundrels.

While the egg begins to hatch a small hut is built on the border of the pit.
Here lives the first priest of the sky serpent. He ensures the health and
wellbeing of the serpent and administers the human sacrifices. Every sacrifice
helps the serpent to grow mightier and stronger, while also improving its
favor.

The residue it produces after every sacrifice fertilizes the fields of the
\gls{Aztecs}, helping their crops grow faster and thereby driving the growth of
the city state.

In its juvenile form the serpent is quite weak. It scales can be penetrated by
hardened weapons thrust with sufficient force. Its magic is limited to casting
a thunderbolt from the sky next to the serpent-priest.

\paragraph{Serpent Rampart}
The pit remains the center of the hamlet while it slowly grows into a village.
To protect the serpent from their enemies the \gls{Aztecs} begin with the
construction of a wall stone and dirt wall surrounding the pit. They also
expand the simple hut into stone quarters, making enough room for two
additional priests.

Once the rampart is completed the serpent begins to feast and grow again. As
its scales harden and its fang lengthen, it becomes a more fearsome monster.
Its magical prowess begins to increase to, permitting the priests to summon
clouds with venom rain. The rain does not kill the afflicted but merely
incapacitates, os they can be taken prisoner and feed to the serpent.

\paragraph{Serpent Fort}
As the village grows into a town the hunger of the serpent grows and the
\gls{Aztecs} prepare accordingly. The simple ramparts are expanded into a fort
fitting for the warriors, the young city state raises. Within the fort the
priestly quarters are expanded to permit five priests to take care of the
serpent.

The cities warriors will raid for prisoners. Beating their enemies unconscious
with heavy wooden clubs. To support them and enable their raids to succeed the
sky-serpents teeth will rip through space and create a two-way corridor between
the fort and a scouted position.

While the serpent can open the corridor at will it can not close it. So the
\gls{Aztecs} have to be sure that they can beat the enemies on the other side
otherwise they may have doomed themselves and their serpent. Especially since
the serpents scales harden to its final strength during this stage, only
penetrable be the strongest weapons.

\paragraph{Serpent Temple}
As the town grows into a city the fort is reconstructed into a temple, the
ceiling closed, only a pair of gates at the top permits access to the serpent.
Nine priests care for it and the eggs it will lay during this stage. The eggs
permitting the expansion of the young kingdom of the \gls{Aztecs}.

With the daily sacrifices the serpent can now bestow the warriors of the
\gls{Aztecs} with a skin of scales. For a short while their warriors skin
becomes as hard as the serpents scales making them immune to most blades and
ranged weapons.

\paragraph{Serpent Pyramid}
At the last stage the temple is completed as a large pyramid. In its center a
giant serpent slowly growing the wings it needs to lift itself up into the sky.
This is the final stage of the temple and the serpent. Deep in the bowls of the
temple a few of the thirteen priests are already nurturing the serpent that
will replace the current one, once it was fed to the sun.

Once the pyramid is complete the hunger of the serpent increases further. To
satisfy it, it can create a corridor between any point its priests can see and
its maw. For a few seconds the corridor will open and all the enemies will fall
straight into its maw permitting it to consumer them.

Once the serpent has fully grown it will leave the pyramid through the gates
and fly towards the sky. During the night it will fly around the pyramid
investigating its surrounding. Once the sun rises the serpent will be devoured
by it strengthening the sun and perpetuating the cycle of life and death.

After the fully grown serpent was consumer, the juvenile at the bottom of the
pyramid will now be fed by the priests until it reaches the same size. Since
the pyramid already exists and the serpent chews quite fast, the only real
limit for the repetition of the cycle is availability of food. So the citizen
of the metropolis and the columns of prisoners will forever ascend the stairs
of the pyramid to fed the serpent so it can fed the ever-dying sun, fighting
back the darkness.

\subsubsection{Pyramid of the ever-dying sun}
The Pyramid of the ever-dying sun is a building as large as the serpent stable.
In its center is a large vessel. Within this vessel the sun-priests collect the
life force of the sun. A few drops every day slowly refilling it.

It would take a long while to fill the vessel just by the daily sun shine, but
whenever a sky serpent is consumed the sun gives enough life-force to fill the
vessel.

So the sun priest can only really work their magic once the serpents stables
have become productive. A especially wealthy metropolis might even have
multiple sun-pyramids to store all the sun-shine generated by the steady
consumption of the sky serpents.

\paragraph{Holy sunshine}
Within a large the life-force of the sun falls as rain from the sky healing the
warriors of the \gls{Aztecs} and burning their enemies.

\paragraph{Blinding Light}
The suns blinding light immobilizes all enemies carrying ranged weapons.

\paragraph{The ray of fire}
The sun sends out a great immolating beam burning through wood, stone and
enemies of the \gls{Aztecs}. No matter how hight the walls are that their
enemies built, this ray will burn them down.

\paragraph{Darkness}
The priests channel the life force of the sun back at her in the morning, so no
life-force is lost during its passage. The \gls{Aztecs} rejoice their warriors
fighting harder while all other people despair. No crops grow unless watered
and fertilized and the wild animals hide in the wilderness. While their enemies
economy crumbles the \gls{Aztecs} lay siege to their cities, beating their
demoralized enemies.

\subsubsection{Forest deities}
Within the forest the \gls{Aztecs} grow avocados, pineapples, tomatoes and
pepper. While they can be grown all over the forest the \gls{Aztecs} prefer to
grow them on special places, encouraging growth with sacrifices to minor
deities, which opposed to the major deities reside within the forest. Their
influence reaches roughly around their sacred place and is rather weak. They
are deities of growth and protection, warding of enemies and increasing the
growth of their favored plants around them.

Usually the monuments of the minor deities and the growing places are combined
into minor forest pantheons next to a gather-hut to permit more easy harvest
and worship by the \gls{Aztecs} in the forest.

\paragraph{The old root}
The old root is majorly concerned with the growth of the trees, especially the
avocado tree. It is worshipped at a great stone steel decorated with obsidian.
Around it the \gls{Aztecs} plant Avocado trees. If the grove is large enough
the old root can use them to ensnare intruders without harming them.

Since it is a deity of the forest and the veins, it despises the Macuahuitl of
the \gls{Aztecs} and the burning mountain. If Macuahuitl are brought to the
stele they are turned back into stone and the trees grow faster.

\paragraph{The sweet embrace}
The sweet embrace is a goddess of delight and deceit. She is worshipped at a
statute representing her. Below the waist she has the body of a serpent, above
the waste the body of a young women. Her head takes the form of a pineapple.
Holy to her are the pineapples and venomous snakes. The snakes can be fed with
fattened eels to gain her favor, in exchange she will sweeten pineapples more
quickly.

While the pineapples offers sweet earthly delights, between them the snakes
wait for intruders. Once bitten they will weaken, the venom slowly destroying
their health if they fail to retreat or destroy her sanctum.

\paragraph{Tomatoes}
The great bloom sees as its domain all the smaller herbs and bushes within the
forest. While the veins are the most numerous its special gift to the
\gls{Aztecs} are tomatoes, growing around its sacred mosaic. It is a cheerful
deity concerned with life and fertility as such it demands nothing more for its
favor than a token offer of a few beans to increase the growth of the tomatoes.

If enough tomatoes grow around it, it will surround nearby buildings with a
layer of protective veins. These veins are harmless to the \gls{Aztecs}, but
will throw their thorns against any intruder that attempts to destroy the
protected buildings.

\paragraph{The screaming face}
The screaming face is a protective deity, ensuring the safety of fools and
children by scaring them away from the forest. To facilitate his function the
\gls{Aztecs} erect large stone heads within the jungle, around wich grows a
large amount of pepper. To facilitate the growth of the pepper the spray the
face of the head with a small amount of their blood. Once the pepper fully
surrounds the stone head the screaming face can create from it a mist that
drives of any invaders approaching it.

\subsection{Starting conditions}
Settlement of the \gls{Aztecs} stat quite simple with the sky-serpent pit, its
resident priest a little fertile land and a bunch of peasants. These peasants
have to build huts, cut down lumber plant trees and multiply to feed the
serpent in the center of their settlement.

\subsection{Play-style}
A giant horde in a buzzing city. The complex diet and the threat of the
serpents hunger keep the \gls{Aztecs} on their toes. Eat good and be not eaten
is a central theme for them. Their people are also integrated into their
economy not only as workers but also as resource. The constant need for human
lives, makes the occasional offensive quite cheap compared to the cost of
running a city of the \gls{Aztecs}.

While their walls offer them safety, they should raid their neighbors to feed
the serpent. A strong and fully upgraded serpent pyramid will make the
\gls{Aztecs} terrible foes, which can annihilate an entire city just to further
grow in power by the very act.

The jungle may seem like a great treasure and an excellent defensive perimeter
until the enemies replace their assault forces by a humble lumberjack and watch
the \gls{Aztecs} despair as the trees they have been nurturing for half an
eternity are turned into charcoal and cheap furniture. Their enemies risk a
lumberjack the \gls{Aztecs} their entire food economy.

The \gls{Aztecs} economy even stretches to the \gls{Aztecs} themselves. People
are sacrifice to create more food and more people slowly snowballing into a
large horde. While other peoples loot cities for the resources within the
\gls{Aztecs} loot them for prisoners, that they can feed into their economy.
For them people are just another resource.


\section{\Glsfmttext{Japanese}}\label{ch:Tribes:Japanses}

\begin{flushright}
	\emph{Balance in all things.}
\end{flushright}
% They have dirt walls, which are harder to destroy 
% They sacrifice food and booze for magic spells
% They are the boring all rounders.


\section{\Gls{Romans}}

\begin{flushright}
	\emph{No god but steel, no laws but ours.}
\end{flushright}
% can consume all the luxury goods of the other people
% Officers can create formations for the warriors around them
% Mighty siege engines
% Classical roman galleys (one) with canons
% Legionnaires
% Light / Heavy metal armor, lances, normal and great helmets, bows, muskets, medium and large shields, javelins etc. 
% They focus on good craftsmen and trade to grow an economy and large population.
% Their theme is industrial might


\newpage{}
\begin{landscape}

	\section{Comparisons}
	For a short overview the tribes are summarized in a few tables.

	\subsection{Overview}
	These table compares the different tribes. They are not precise measurement but
	should help give a rough oversight. The entries range from \textquote{lowest}
	\textquote{lower} \textquote{low} \textquote{medium} \textquote{high}
	\textquote{higher} and \textquote{highest}.

	\begin{longtable}{p{3cm}p{1.5cm}p{1.5cm}p{1.5cm}p{1.5cm}p{1.8cm}p{6.2cm}}
		\toprule{}
		Tribe
		  & Area coverage\footnote{
			This refers to the area necessary to build up the industry relative to
			other tribes.
		} & Mobility
		  & Defensive strength                                      & Offensive strength
		  & Economy complexity
		  & Unique characteristic                                                        \\
		\midrule{}
		The \gls{Vikings}
		  & Highest                                                 & Medium
		  & Higher                                                  & Medium
		  & Hight
		  & Natural sanctuaries enforce environmental adoption
		\\
		The \gls{Nomads}
		  & Higher                                                  & Highest
		  & Lowest                                                  & High
		  & Lowest
		  & Completely mobile except for their temples
		\\
		The \gls{Egyptians}
		  & Lowest                                                  & Lowest
		  & Highest                                                 & Lower
		  & High
		  & Able to reanimate fallen soldiers
		\\
		The \gls{Aztecs}
		  & Lower                                                   & Low
		  & Lower                                                   & Lower
		  & Higher
		  & They participate in their economy as a resource
		\\
		The \gls{Japanese}
		  & Medium                                                  & Medium
		  & Medium                                                  & Medium
		  & Medium
		  & None
		\\
		The \gls{Romans}
		  & Higher                                                  & Medium
		  & Low                                                     & Highest
		  & Highest
		  & Atheists able to use the luxuries from all other tribes
		\\
		\bottomrule{}
	\end{longtable}

	\newpage

	\subsection{Weapon technology}
	Weapon technology comes to the tribes in three major ways. The ability to
	fortify their cities, the ability to work complex metals and their ability to
	build projectile weapons.

	\begin{longtable}{p{5cm}p{5cm}p{4cm}p{6cm}}
		\toprule{}
		Tribe
		 & Fortifications
		 & Metallurgy
		 & Projectile Weapons                \\
		\midrule{}
		The \gls{Vikings}
		 & Palisades, Stone walls and towers
		 & Copper, Iron
		 & Short Bow                         \\
		The \gls{Nomads}
		 & Wooden towers
		 & None
		 & Short Bow                         \\
		The \gls{Egyptians}
		 & High stone walls and towers
		 & Copper
		 & Short Bow                         \\
		The \gls{Aztecs}
		 & Earthen ramparts
		 & None
		 & Atlatls, Slings                   \\
		\bottomrule{}
	\end{longtable}

	\subsection{Biome preference}
	The different tribes have different advantages while settling in the different
	\hyperref[sec::biomes]{biomes}. These are listed here.

	\begin{longtable}{p{5cm}ccccc}
		\toprule{}
		Tribe
		 & Grassland
		 & Forest
		 & Jungle
		 & Desert
		 & Mountain  \\
		\midrule{}
		The \gls{Vikings}
		 & Higher
		 & Higher
		 & Lower
		 & Lowest
		 & Higher    \\
		The \gls{Nomads}
		 & Highest
		 & Lower
		 & Lower
		 & Lowest
		 & Lowest    \\
		The \gls{Egyptians}
		 & Low
		 & Lower
		 & Lowest
		 & Highest
		 & Medium    \\
		The \gls{Aztecs}
		 & High
		 & High
		 & Higher
		 & Lowest
		 & Lowest    \\
		\bottomrule{}
	\end{longtable}

\end{landscape}


\printglossary[type=tribe, title={Glossary}]{}\label{ch:Tribes:Overview}


\chapter{Goods}
This chapter describes the different goods produced and gathered within the
world.

\section{Natures bounty}
The world is rich in countless natural resources the tribes can use. They can
be categorized in wild animals, plants and minerals. These are described in
this section.

\subsection{What grows}
The world is home to a wide variety of plants. Plants can be categorized into
plants of special interesting like trees or herbs and abundant plants. Abundant
usually cover large areas and are to abundant to track. \emph{Grass} is a good
example. In the following list all special plants will be listed with the
\emph{fruit} they create continuously and the resources they can be
\emph{harvested} for.

\begin{longtable}{llllll}
	\toprule
	Plant         & Fruit    & Harvest  & Biome & Growth & Abundance \\
	\midrule
	Herb          & None     & Herbs    & Forest            & High   & Low       \\
	Mushroom      & None     & Mushroom & Forest            & Medium & Low       \\
	Fir           & None     & Timber   & Forest            & Low    & High      \\
	Oak           & None     & Timber   & Forest, Grassland & Lowest & Lower     \\
	Apple-tree    & Apple    & Timber   & Forest, Grassland & Lower  & Lower     \\
	Chestnut-tree & Chestnut & Timber   & Forest, Grassland & Lower  & Lower     \\
	\bottomrule
\end{longtable}

\paragraph{Grass}
Grass covers the ground within the forests and the grasslands. Some livestock
needs to graze to live. This can only happen on grassland. The grass grows fast
enough so it is not affected by the grazing.

\paragraph{Herbs}
Within the forest medicinal herbs grow. These are can also be used as a food
additive.

\paragraph{Mushromm}
Mushrooms grow within the forest and can be used for medical purposes or as
food.

\paragraph{Fir}
The fir grows within the forest and can be felled for timber.

\paragraph{Oak}
Oak trees dot the forest and grassland and can be felled for timber.

\paragraph{Apple-tree}
Within the grassland and the forest the fruit of apple trees appear, unless
they are cut down for timber.

\paragraph{Chestnut-tree}
The Chestnuts grow in the grassland and the forest and are often consumed by
wild animals, but some of the tribe appreciate them too, others just see
valuable building material in them.

\subsection{What walks}
Within the world there are countless animals with different behavior, uses and
habitats. These are described here.

\begin{longtable}{ccccc}
	\toprule
	Animal & Uses       & Biome & Danger & Speed  \\
	\midrule
	Deer   & Meat, Hide & Forest            & None   & Higher \\
	Boar   & Meat, Hide & Grassland, Forest & Low    & Medium \\
	\bottomrule
\end{longtable}

\paragraph{Deer}
Deer are forest dwellers. They pose no danger to the tribes and flee, when
spotted or attacked. They are hunted for meat and hides.

\paragraph{Boars}
Boars dwell on the grassland and the forest. They are hunted for their meat and
hides and often fight back against the hunters.

\subsection{What swims}
The seas teem with life. The fish within them are caught by the fishers for
sustenance.

\subsection{What stands unmoved}
Within the worlds above the ground valuable resources often appear agglomerated
within nodes. From these nodes they can be mined. How \emph{hard} they are to
mine and how \emph{abundant} they are is listed here.

\begin{longtable}{ccc}
	\toprule
	Mineral & Hardness & Abundance \\
	\midrule
	Rock    & Low      & Highest   \\
	Copper  & Low      & Low       \\
	Iron    & Higher   & Low       \\
	Gold    & Lower    & Lowest    \\
	Coal    & Lowest   & High      \\
	\bottomrule
\end{longtable}

\paragraph{Rock}
Rock is mined for stone and then used as a building material.

\paragraph{Copper}
Copper is the softest metal used for weapons, tools and armor.

\paragraph{Iron}
Iron is the medium metal for weapons, tools and armor. It can also be refined
into steel.

\paragraph{Gold}
Gold is a rare metal used for luxury items and sometimes buildings.

\paragraph{Coal}
Coal can be mined and used for smelting metals.

\section{Armory}
As the inhabitants of this world are rather war-like, the various implements of
conflict are listed here. In general the quality of the devices depends on the
materials used for them. For a better overview the materials can be found
within this table.

\begin{longtable}{cccc}
	\toprule
	Material & Armor-Tier & Shield-Tier & Melee-Tier \\
	\midrule
	Leather  & 1          & 0           & None       \\
	Wood     & None       & 1           & 0          \\
	Copper   & 2          & 2           & 2          \\
	Iron     & 3          & 3           & 3          \\
	Steel    & 4          & 4           & 4          \\
	\bottomrule
\end{longtable}

\subsection{Weapons}
This section lists the various weapons and their qualities. They are
categorized according to the \emph{damage} they do to opponents without armor,
the ease with which they \emph{penetrate} armor, the \emph{speed} they permit
their users attacks, the \emph{training} necessary to employ them efficiently
and their \emph{range}.

\begin{longtable}{lllllr}
	\toprule
	Weapon
	       & Damage  & Penetration
	       & Speed   & Training    & Range \\
	\midrule
	Spear
	       & Low     & Medium
	       & Medium  & Low         & 2 m   \\
	Short Sword
	       & Medium  & Medium
	       & Higher  & High        & 1 m   \\
	Battle Axe
	       & Highest & Lower
	       & Medium  & Lower       & 1 m   \\
	Short bow
	       & Medium  & Lower       &
	Medium & High    & 30 m                \\
	\bottomrule
\end{longtable}

Weapons do not only influence the physical battle but also the mental fortitude
of the enemy. To strike \emph{fear} into their enemies and improve their own
\emph{morale} some tribes decorate their weapons- The following list summarizes
the effect of the decorations.

\begin{longtable}{l ll ll ll}
	\toprule
	Weapon
	 & \multicolumn{3}{c}{Morale}
	 & \multicolumn{3}{c}{Fear}
	\\
	 & Plain                      & Simple & Elegant
	 & Plain                      & Simple & Elegant \\
	\midrule
	Spear
	 & Lowest                     & Lower  & Low
	 & Lowest                     & Lower  & Low     \\
	Short Sword
	 & Lower                      & Medium & High
	 & Lowest                     & Lower  & Low     \\
	Battle Axe
	 & Medium                     & High   & Higher
	 & Lower                      & Medium & High    \\
	Short bow
	 & Lower                      & Low    & Medium
	 & Lowest                     & Lowest & Lowest  \\
	\bottomrule
\end{longtable}

\paragraph{Spear}
Spears or sharpened sticks are ancient weapons. A skilled fighter can hit the
weak spots of enemy armor with it. Its ease of production and long range make
it a quite common weapon for the tribes that employ it.

\paragraph{Sword}
Sword or longer knives are very common weapons. Long swords fare better against
lighter armoured opponents, while short swords offer a good compromise for
melee weapons.

\paragraph{Battle axe}
The battle axe is a brute instrument that excels against unarmed opponents. The
ease of use and devastating effect make it the favorite weapon of the
\gls{Vikings}.

\paragraph{Bow}
Bows are acceptable ranged-weapons. They do better against lightly armored
enemies and have an acceptable range.

\subsection{Armor}
To protect them selves against their opponents the tribes fabricate different
forms of armor and defenses. Those are listed here and categorized according to
the protection they offer against \emph{close ranged} weapons, like spears and
swords, the protection they offer against \emph{projectile} weapons like bows.
In addition the \emph{weight} inhibiting the wearers movement and the effect on
the wearers \emph{morale} are listed.

\begin{longtable}{lllll}
	\toprule
	Equipment
	 & Close range & Projectile & Weight  & Morale  \\
	\midrule
	Light body-armor
	 & Medium      & Low        & Low     & Low     \\
	Medium body-armor
	 & Higher      & Medium     & High    & Medium  \\
	Heavy body-armor
	 & Highest     & High       & Highest & High    \\
	\midrule
	Light shield
	 & Lowest      & Low        & Lowest  & Lowest  \\
	Medium shield
	 & Lower       & High       & Lower   & Lower   \\
	Heavy shield
	 & Medium      & Highest    & Low     & Low     \\
	\midrule
	Light helmet
	 & Lowest      & Lowest     & Lowest  & High    \\
	Medium helmet
	 & Lower       & Lowest     & Lower   & Higher  \\
	Heavy helmet
	 & Low         & Lowest     & Low     & Highest \\
	\bottomrule
\end{longtable}

Decorations that are added by some tribes to the pieces of armor have different
effects on the morale of the wearer. No equipment will lower the morale of the
wearer but the influence of the different values of elaboration are listed
here.

\begin{longtable}{llll}
	\toprule
	Category   & Plain  & Simple & Elegant \\
	\midrule
	Body-armor & Lower  & Low    & Medium  \\
	Shield     & Lowest & Lower  & Low     \\
	Helmet     & Lower  & High   & Highest \\
	\bottomrule
\end{longtable}

\paragraph{Body armor}
Body armor is primarily supposed to protect the wearer from harm. It is usually
categorized into light, medium and heavy according to hiw its weight and the
mobility it permits its wearer. The decorated variations are more prestigious,
instilling in the wearers the desire to act braver than they would otherwise.

\paragraph{Shield}
Shields are mostly used to protect against projectile weapons, were they excel.
Like body armor they maybe decorated to improve the warriors morale.

\paragraph{Helmets}
While helmets protect the heads of the wearers. They are often richly decorated
convening prestige and rank. In general a fancy hat means a warrior stays in
battle longer, before they route.


% # Notes
% - Fortifications should be slow to build and tear down, so they feel more meaningful
% and siege magic more mighty.
% - It should be possible to configure the equipment of soldiers in the form of templates
% preferable on the individual level
% - Having the option to start with customized starting towns/fortresses would permit player
% to create a strategy up ahead. They might invest in an early keep or a large weapons industry
% permitting them to rush. This should make matches more varied for experienced players.
% 
% Considering how much I enjoyed building and walking through cities in Spell force.
% The campaign could be made similar. 
% As an immortal the player character traverses the world recruiting randomly generated npcs.
% Between the maps he only can take a limited number of people with him through the gate or in a ship.
%
% The people he leaves behind keep living their lives and the cities change and grow a little while the AI manages them. 
% 
% If the player built the cities well he returns to flourishing lands and can recruit the children of his former followers into his retinue to fight the next battle and shop for good equipment.
% 
% He can also trade with these cities.
% This will result in a player created interactive world with mini quests and make the changes and threats more meaningful.
% 
% It is in some sense like X4 just with more RTS RPG elements.
% (Forges leveling up overtime giving better equipment, supply and demand, legions of soldiers to recruit, blueprints for buildings to buy, new cultures to encounter etc.)
% 
% Food production also ties into this as the retinues hunger makes a settlement necessary and the available farmland limits the size of the final settlement.

\end{document}
