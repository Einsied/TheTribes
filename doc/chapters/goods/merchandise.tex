\section{\Glsfmttext{Merchandise}}\label{ch:Goods:Merchandise}

The members of the \hyperref[ch:Tribes]{tribes} enjoy various articles of
\gls*{Merchandise}, like clothes or furniture. These can be categorized
according to their desirability or quality, need and their durability.
Durability describes how fast an item is worn out and needs to be replaced or
repaired. While commonly durability is thought of as the degradation of a
physical item here it also describes the consumption of an item like a
\gls{Drink}. So a mug of ale reaches the end of its durability, when a new one
is desired.

\Gls{Merchandise} that has reached the end of its durability is discarded.
In some cases this creates a \emph{used} item of \gls{Merchandise} that can be recycled.

Considering that the \hyperref[ch:Tribes]{tribes} produce a larger variety of
\gls*{Merchandise}, first the needs are described and the targets of
distribution. \Gls{Clothes} is used by an individual, while \gls{Furniture} can
be shared between multiple inhabitants of a home, so \Gls{Clothes} is
distributed to individuals while \gls{Furniture} is distributed to homes. Some
goods may provide for more than one need and the strength of the needs varies
between the \hyperref[ch:Tribes]{tribes} and their members.

\begin{longtable}{ccc}
	\toprule
	Need             & Distribution & \Gls{Merchandise} \\
	\midrule
	\Gls{Adornments} & Individual   & \Gls{Jewlery}     \\
	\Gls{Clothing}   & Individual   & \Gls{Clothes}     \\
	\Gls{Comfort}    & Home         & \Gls{Furniture}   \\
	\Gls{Carousing}  & Tavern       & \Gls{Drink}       \\
	\bottomrule
\end{longtable}

\subsection{\Glsfmttext{Jewlery}}\label{ch:Goods:Merchandise:Jewlery}

\Gls{Jewlery} is made primary made from different materials and in different qualities by the
\hyperref[ch:Tribes]{tribes}. Here is a short summary:

\begin{longtable}{ccc}
	\toprule
	\Gls*{Jewlery} & Quality & Producer      \\
	\midrule
	% Todo create a good for viking gold jewelry
	Golden rings   & Higher  & \Gls{Vikings} \\ \bottomrule
\end{longtable}

\subsection{\Glsfmttext{Clothes}}\label{ch:Goods:Merchandise:Clothes}

\Gls{Clothes} is made primary made from different materials and in different qualities by the
\hyperref[ch:Tribes]{tribes}. Here is a short summary:

\begin{longtable}{ccc}
	\toprule
	\Gls*{Clothes} & Quality     & Producer    \\
	\midrule
	Placeholder    & Placeholder & Placeholder \\
	\bottomrule
\end{longtable}

\subsection{\Glsfmttext{Furniture}}\label{ch:Goods:Merchandise:Furniture}

\Gls{Furniture} is made primary made from wood, but its produced in different qualities by the
\hyperref[ch:Tribes]{tribes}. Here is a table summarizing the results:

\begin{longtable}{ccc}
	\toprule
	\Gls{Furniture} & Quality     & Producer    \\
	\midrule
	Placeholder     & Placeholder & Placeholder \\
	\bottomrule
\end{longtable}
