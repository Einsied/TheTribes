\section{Armory}\label{ch:Goods:Armory}

As the inhabitants of this world are rather war-like, the various implements of
conflict are listed here. In general the quality of the devices depends on the
materials used to produce them. For a better overview the materials and
expected quality are listed in a table.

\begin{longtable}{lllll}
	\toprule
	Material   & \Gls{BodyArmor} & \Gls{Helmet} & \Gls{Shield} & \Gls{Weapon} \\
	\midrule
	Leather    & Lowest          & Lowest       & Lowest       & Lowest       \\
	Rock       & None            & None         & None         & Lower        \\
	Wood       & None            & None         & Lower        & Lower        \\
	Obsidian   & None            & None         & None         & Low          \\
	Copper     & Low             & Low          & Low          & Low          \\
	Iron       & Medium          & Medium       & Medium       & Medium       \\
	Steel      & High            & High         & High         & High         \\ % Place for further expansion of metals
	Moonsilver & Highest         & Highest      & Highest      & Highest      \\
	\bottomrule
\end{longtable}

Implements of war are also often decorated to strengthen the morale of the user
or strike fear into their opponents if they are \glspl{Weapon}. The amount of
decoration is usually categorized as \emph{plain}, \emph{simple} or
\emph{elegant}.

Using the kind of implement the material and the decoration permits a rough
categorization and first impression, that can later be refined by a look at the
details.

\subsection{\Glsfmttext{Weapon}s}\label{ch:Goods:Armory:Weapons}

Various weapons are employed by the
\hyperref[ch:World:Inhabitants:Sapients]{sapient inhabitants} of this world,
most of them manufactured and used by the \hyperref[ch:Tribes]{tribes} in
\hyperref[chch:Conflict:Combat]{combat}. Since there is a great number of
different weapons they are categorized in more abstract terms.

For a first overview we compare how much \hyperref[ch:Conflict:Combat]{harm}
the \glspl{Weapon} do, how well they penetrate \gls{Armor} and how far they
reach\footnote{ Range is different for different weapon types. So a melee
	weapon with \emph{lowest} range might reach 1 meter, but a ranged weapon with
	\emph{lowest} range might reach 10 meter. }.

\begin{longtable}{lllllll}
	\toprule
	Weapon
	 & Physical & Penetration
	 & Stun     & Morale
	 & Range    & Ranged      \\
	\midrule
	Battle Axe
	 & Highest  & Lower
	 & Lower    & Medium
	 & Lowest   & No          \\
	Short sword
	 & Medium   & Medium
	 & Lower    & Lower
	 & Lowest   & No          \\
	Spear
	 & Low      & Medium
	 & Lowest   & Lower
	 & Lower    & No          \\
	Short bow
	 & Medium   & Lower
	 & Lower    & Lower
	 & Medium   & Yes         \\
	\bottomrule
\end{longtable}

Next we consider how the do \hyperref[ch:Conflict:Combat]{harm} on the
battlefield. First we took a look at the direct physical results of the
\glspl{Weapon}.

\begin{longtable}{llll}
	\toprule
	Weapon
	 & \multicolumn{3}{c}{Physical and Stun}                              \\
	 & Distribution                          & Delivery   & Avoids allies \\
	\midrule
	Battle Axe
	 & Single                                & Direct     & Yes           \\
	Short sword
	 & Single                                & Direct     & Yes           \\
	Spear
	 & Single                                & Direct     & Yes           \\
	Short bow
	 & Single                                & Projectile & No            \\
	\bottomrule
\end{longtable}

The next table summarizes how \hyperref[ch:Conflict:Combat]{harm} is done to
the spirit or morale of the opponent.

\begin{longtable}{llll}
	\toprule
	Weapon
	 & \multicolumn{3}{c}{Morale}                               \\
	 & Distribution               & Delivery   & Discrimination \\
	\midrule
	Battle Axe
	 & Area                       & Direct     & Yes            \\
	Short sword
	 & Area                       & Direct     & Yes            \\
	Spear
	 & Area                       & Direct     & Yes            \\
	Short bow
	 & Area                       & Projectile & No             \\
	\bottomrule
\end{longtable}

Lastly we have to think about how the \glspl{Weapon} affect their users. This
include the \emph{speed} at which it can be used, the \emph{training} required
to use it, the \emph{morale} boost the user experiences and lastly the
\emph{weight} on the user. \emph{Weight} is more relevant for armor an limits
the users speed overall.

\begin{longtable}{lllll}
	\toprule
	Weapon
	 & Speed  & Training
	 & Morale & Weight   \\
	\midrule
	Battle Axe
	 & Medium & Low
	 & Medium & High     \\
	Short sword
	 & Higher & High
	 & Medium & Lower    \\
	Spear
	 & Medium & Low
	 & Lower  & Low      \\
	Short bow
	 & Medium & High
	 & Low    & Lower    \\
	\bottomrule
\end{longtable}

\subsubsection{Spear}\label{ch:Goods:Armory:Weapons:Spear}

Spears or sharpened sticks are ancient weapons. A skilled fighter can hit the
weak spots of enemy armor with it. Its ease of production and long range make
it a quite common weapon for the tribes that employ it.

\subsubsection{Sword}\label{ch:Goods:Armory:Weapons:Sword}

Sword or longer knives are very common weapons. Long swords fare better against
lighter armoured opponents, while short swords offer a good compromise for
melee weapons.

\subsubsection{Battle axe}\label{ch:Goods:Armory:Weapons:BattleAxe}

The battle axe is a brute instrument that excels against unarmed opponents. The
ease of use and devastating effect make it the favorite weapon of the
\gls{Vikings}.

\paragraph{Bow}

Bows are acceptable ranged-weapons. They do better against lightly armored
enemies and have an acceptable range.

\subsection{\Glsfmttext{Armor}}\label{ch:Goods:Armory:Armor}

To protect them selves against their opponents the tribes fabricate different
forms of \gls*{Armor} and defenses. Those can be categorized according to the
protection they offer against melee/\hyperref[ch:Conflict]{\emph{direct}} or
\hyperref[ch:Conflict]{\emph{projectile}} \glspl{Weapon}, the \emph{weight}
inhibiting the wearers movement and the effect on the wearers \emph{morale}.

\begin{longtable}{llllll}
	\toprule
	\Gls*{Armor}
	 & Close range & Stun   & Projectile & Weight & Morale \\
	\midrule
	\Gls{BodyArmor}
	 & Higher      & Medium & Medium     & High   & Medium \\
	\Gls{Shield}
	 & Lower       & Lower  & High       & Lower  & Lower  \\
	\Gls{Helmet}
	 & Lower       & Medium & Lowest     & Lower  & Higher \\
	\bottomrule
\end{longtable}

In contrast to \glspl{Weapon} \gls*{Armor} is not sub-categorized into types
but into weight classes. So every item of armor is classed as \emph{light},
\emph{medium} and \emph{heavy} in addition to the three decoration categories,
so we may encounter a \textquote{Heavy elegant \gls{Helmet}} on the
battlefield, which should have highest values among most \glspl{Helmet}.

The level of decoration and weight influences the values for the pieces of
\gls{Armor} with different strengths. So a \emph{heavy} \gls{BodyArmor} is
heavier than a \emph{medium} one relative to a \emph{heavy} and \emph{medium}
\gls{Helmet}. So the influence ot the weight is tracked here:

\begin{longtable}{llllll}
	\toprule
	\Gls*{Armor}
	 & Close range & Stun   & Projectile & Weight & Morale \\
	\midrule
	\Gls{BodyArmor}
	 & Medium      & Low    & Medium     & Higher & Low    \\
	\Gls{Shield}
	 & Lower       & Lower  & High       & Lower  & Lower  \\
	\Gls{Helmet}
	 & Lowest      & Medium & Lowest     & Lowest & Low    \\
	\bottomrule
\end{longtable}

Decorations that are added by some \hyperref[ch:Tribes]{tribes} to the pieces
of \gls*{Armor} have different effects on the morale of the wearer. No piece of
\gls*{Armor} will lower the morale of the wearer but the influence differs
significantly.

\begin{longtable}{ll}
	\toprule
	\Gls*{Armor}    & Decoration-Impact \\
	\midrule
	\Gls{BodyArmor} & Low               \\
	\Gls{Shield}    & Low               \\
	\Gls{Helmet}    & High              \\
	\bottomrule
\end{longtable}

\subsubsection{\Glsfmttext{BodyArmor}}\label{ch:Goods:Armory:Armor:BodyArmor}

\Gls*{BodyArmor} is primarily supposed to protect the wearer from harm. It is usually
categorized into light, medium and heavy according to its weight and the
mobility it permits its wearer. The decorated variations are more prestigious,
instilling in the wearers the desire to act braver than they would otherwise.

\subsubsection{\Glsfmttext{Shield}}\label{ch:Goods:Armory:Armor:Shield}

\Glspl*{Shield} are mostly used to protect against projectile weapons, were they excel.
They present a tradeoff, because the hand carrying the \gls*{Shield}
Like \gls{BodyArmor} they maybe decorated to improve the warriors morale.

\subsubsection{\Glsfmttext{Helmet}}\label{ch:Goods:Armory:Armor:Helmet}

A \gls*{Helmet} protects the head of the wearers from decapitating blows and
stun-damage. They are often richly decorated convening prestige and rank. In
general a fancy hat means a warrior stays in battle longer, before they route
or fall unconscious.
