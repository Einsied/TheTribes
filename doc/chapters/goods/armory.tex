\section{Armory}\label{ch:Goods:Armory}

As the inhabitants of this world are rather war-like, the various implements of
conflict are listed here. In general the quality of the devices depends on the
materials used for them. For a better overview the materials can be found
within this table.

\begin{longtable}{llll}
	\toprule
	Material   & Armor-Tier & Shield-Tier & Melee-Tier \\
	\midrule
	Leather    & 1          & 0           & None       \\
	Wood       & None       & 1           & 0          \\
	Copper     & 2          & 2           & 2          \\
	Iron       & 3          & 3           & 3          \\
	Steel      & 4          & 4           & 4          \\ % Place for further expansion of metals
	Moonsilver & 7          & 6           & 6          \\
	\bottomrule
\end{longtable}

\subsection{\Glsfmttext{Weapon}s}\label{ch:Goods:Armory:Weapons}

This section lists the various weapons and their qualities. They are
categorized according to the \emph{damage} they do to opponents without armor,
the ease with which they \emph{penetrate} armor, the \emph{speed} they permit
their users attacks, the \emph{training} necessary to employ them efficiently
and their \emph{range}.

\begin{longtable}{lllllr}
	\toprule
	Weapon
	 & Damage  & Penetration
	 & Speed   & Training    & Range \\
	\midrule
	Spear
	 & Low     & Medium
	 & Medium  & Low         & 2 m   \\
	Short Sword
	 & Medium  & Medium
	 & Higher  & High        & 1 m   \\
	Battle Axe
	 & Highest & Lower
	 & Medium  & Lower       & 1 m   \\
	Short bow
	 & Medium  & Lower
	 & Medium  & High        & 30 m  \\
	\bottomrule
\end{longtable}

Weapons do not only influence the physical battle but also the mental fortitude
of the enemy. To strike \emph{fear} into their enemies and improve their own
\emph{morale} some tribes decorate their weapons- The following list summarizes
the effect of the decorations.

\begin{longtable}{l ll ll ll}
	\toprule
	Weapon
	 & \multicolumn{3}{c}{Morale}
	 & \multicolumn{3}{c}{Fear}
	\\
	 & Plain                      & Simple & Elegant
	 & Plain                      & Simple & Elegant \\
	\midrule
	Spear
	 & Lowest                     & Lower  & Low
	 & Lowest                     & Lower  & Low     \\
	Short Sword
	 & Lower                      & Medium & High
	 & Lowest                     & Lower  & Low     \\
	Battle Axe
	 & Medium                     & High   & Higher
	 & Lower                      & Medium & High    \\
	Short bow
	 & Lower                      & Low    & Medium
	 & Lowest                     & Lowest & Lowest  \\
	\bottomrule
\end{longtable}

\subsubsection{Spear}\label{ch:Goods:Armory:Weapons:Spear}

Spears or sharpened sticks are ancient weapons. A skilled fighter can hit the
weak spots of enemy armor with it. Its ease of production and long range make
it a quite common weapon for the tribes that employ it.

\subsubsection{Sword}\label{ch:Goods:Armory:Weapons:Sword}

Sword or longer knives are very common weapons. Long swords fare better against
lighter armoured opponents, while short swords offer a good compromise for
melee weapons.

\subsubsection{Battle axe}\label{ch:Goods:Armory:Weapons:BattleAxe}

The battle axe is a brute instrument that excels against unarmed opponents. The
ease of use and devastating effect make it the favorite weapon of the
\gls{Vikings}.

\paragraph{Bow}

Bows are acceptable ranged-weapons. They do better against lightly armored
enemies and have an acceptable range.

\subsection{\Glsfmttext{Armor}}\label{ch:Goods:Armory:Armor}

To protect them selves against their opponents the tribes fabricate different
forms of \gls{Armor} and defenses. Those are listed here and categorized
according to the protection they offer against \emph{close ranged} weapons,
like spears and swords, the protection they offer against \emph{projectile}
weapons like bows. In addition the \emph{weight} inhibiting the wearers
movement and the effect on the wearers \emph{morale} are listed.

\begin{longtable}{lllll}
	\toprule
	Equipment
	 & Close range & Projectile & Weight  & Morale  \\
	\midrule
	Light \gls{BodyArmor}
	 & Medium      & Low        & Low     & Low     \\
	Medium \gls{BodyArmor}
	 & Higher      & Medium     & High    & Medium  \\
	Heavy \gls{BodyArmor}
	 & Highest     & High       & Highest & High    \\
	\midrule
	Light \gls{Shield}
	 & Lowest      & Low        & Lowest  & Lowest  \\
	Medium \gls{Shield}
	 & Lower       & High       & Lower   & Lower   \\
	Heavy \gls{Shield}
	 & Medium      & Highest    & Low     & Low     \\
	\midrule
	Light \gls{Helmet}
	 & Lowest      & Lowest     & Lowest  & High    \\
	Medium \gls{Helmet}
	 & Lower       & Lowest     & Lower   & Higher  \\
	Heavy \gls{Helmet}
	 & Low         & Lowest     & Low     & Highest \\
	\bottomrule
\end{longtable}

Decorations that are added by some tribes to the pieces of armor have different
effects on the morale of the wearer. No equipment will lower the morale of the
wearer but the influence of the different values of elaboration are listed
here.

\begin{longtable}{llll}
	\toprule
	Category        & Plain  & Simple & Elegant \\
	\midrule
	\Gls{BodyArmor} & Lower  & Low    & Medium  \\
	\Gls{Shield}    & Lowest & Lower  & Low     \\
	\Gls{Helmet}    & Lower  & High   & Highest \\
	\bottomrule
\end{longtable}

\subsubsection{\Glsfmttext{BodyArmor}}\label{ch:Goods:Armory:Armor:BodyArmor}

\Gls*{BodyArmor} is primarily supposed to protect the wearer from harm. It is usually
categorized into light, medium and heavy according to its weight and the
mobility it permits its wearer. The decorated variations are more prestigious,
instilling in the wearers the desire to act braver than they would otherwise.

\subsubsection{\Glsfmttext{Shield}}\label{ch:Goods:Armory:Armor:Shield}

\Glspl*{Shield} are mostly used to protect against projectile weapons, were they excel.
They present a tradeoff, because the hand carrying the \gls*{Shield}
Like \gls{BodyArmor} they maybe decorated to improve the warriors morale.

\subsubsection{\Glsfmttext{Helmet}}\label{ch:Goods:Armory:Armor:Helmet}

A \gls*{Helmet} protects the head of the wearers from decapitating blows and
stun-damage. They are often richly decorated convening prestige and rank. In
general a fancy hat means a warrior stays in battle longer, before they route
or fall unconscious.
