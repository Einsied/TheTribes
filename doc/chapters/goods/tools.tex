\section{\Glsfmttext{Tool}}\label{ch:Goods:Tools}

Most works by members of the \hyperref[ch:Tribes]{tribes} requires
\glspl{Tool}. Some work can also done with improvised tools, which are created
on the spot by the craftsmen. These improvised tools do not participate in the
economy. \Glspl{Tool} break after a certain number of uses depending on their
quality. If a tool is broken the craftsmen will conduct some emergency repairs
and keep working with the broken \gls{Tool} until a new \gls{Tool} is
delivered.

Work done with improvised or broken \glspl{Tool} is significantly slower than
work done with new ones. Besides this there is no difference between the
\glspl{Tool}, so work done with an iron \gls{Pick} is just as fast as one with
a steel \gls{Pick}. The iron one just breaks faster.

Broken \glspl{Tool} can often be recycled to regain the original materials or
directly repaired.

\subsection{\Glsfmttext{Pick}}

\Glspl{Pick} are made from metal or other hard materials and are often used for mining.
The \hyperref[ch:Tribes]{tribes} produce them from different materials and with different
qualities.
Here is a short summary:

\begin{longtable}{ccc}
	\toprule
	\Gls*{Pick} & Quality     & Producer    \\
	\midrule
	Placeholder & Placeholder & Placeholder \\
	\bottomrule
\end{longtable}
