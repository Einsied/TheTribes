\chapter{Conflict}\label{ch:Conflict}

While the \hyperref[ch:World:Inhabitants:Animals]{animals} and the none
organized \hyperref[ch:World:Inhabitants:Sapients]{sapients} live in blissful
harmony the members of the \hyperref[ch:Tribes]{tribes} do not. This leads to
conflict, which often takes the form of combat.

\section{Direct combat}\label{ch:Conflict:Combat}

Direct combat refers to a fight between multiple
\hyperref[ch:World:Inhabitants] of the world. While members of the
\hyperref[ch:Tribes]{tribes} usually use \glspl{Weapon} to do harm to each
other \hyperref[ch:World:Inhabitants:Animals]{animals} or
\hyperref[ch:World:Inhabitants:Livestock]{livestock} use their bodies, hooves,
claws and teeth.

There is more than one way to do harm and the results are different. The
methods to \hyperref[ch:Goods:Armory:Armor]{avoid harm} or recover from it are
also different. In general all harm gets slowly undone, but some can be cured
faster with special supplies. Once there is more harm done than can be absorbed
the \emph{final effect}, like death, occurs. Here is a short overview:

\begin{longtable}{llll}
	\toprule
	Harm     & Recovery speed & Recovery \hyperref[ch:Goods]{goods} & Final Effect                                                                               \\
	\midrule
	Physical & Lowest         & \Gls{Medicine}                      & Death                                                                                      \\
	Morale   & Faster         & \Gls{Drink}                         & Panic\footnote{The affected creatures flees the battle if it can.}                         \\
	Stun     & Medium         & None                                & Loss of consciousness\footnote{This permits the creatures opponent to capture or kill it.} \\
	\bottomrule
\end{longtable}

Harm can also be distinguished by the way it is distributed. A blade strikes
one opponent directly, an arrow travels to its target and a comrade being split
in two strikes fear in all around.

\begin{longtable}{ll}
	\toprule
	Category       & Description                                             \\
	\midrule
	Distribution   & If the harm only affects a single target or multiple.   \\
	Delivery       & If the harm is applied directly or via an projectile.   \\
	Discrimination & If the harm discriminates between opponents and allies. \\
	\bottomrule
\end{longtable}

\section{Siege warfare}\label{ch:Conflict:Sieges}

Siege-warfare refers to fight between members of the
\hyperref[ch:Tribes]{tribes} and fortifications often manned by members of
other \hyperref[ch:Tribes]{tribes}. Here we will discuss how the different kind
of fortifications influence \hyperref[ch:Conflict:Combat]{combat} and how they
can be destroyed or their advantages neutralized by siege-engines.
