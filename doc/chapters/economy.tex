\chapter{Economy}\label{ch:Economy}

To thrive the \hyperref[ch:Tribes]{tribes} need to produce different
\hyperref[ch:Good]{goods}. Their production can be roughly divided into the
gathering of \hyperref[ch:Goods:Nature]{raw inputs}, their refinement and use
or consumption. This means \hyperref[ch:Goods]{goods} are first gathered then
transported along a production-chain, before they reach a final destination for
use or consumption.

\section{Production}\label{ch:Economy:Production}

Production occurs at production-sites. For resource gathering operations these
are the centers of work. So a \glslink{CopperOre}{copper}-miner will grab their
\gls{Pick} from the mine and walk to the next resource node containing
\gls{CopperOre}. After mining the ore he will then deliver the \gls{CopperOre}
to the mine, were it is stored in the mines output area.

If the \gls{CopperOre} has to be transformed into a \gls{Tool} it is
transported to the \gls{CopperOre}-input of a smithy. There the smith
transforms it into a \gls{Tool} using \gls{Coal} from the \gls{Coal}-input.
Then the smith places it in the output-area of the smithy.

This means that all production-sites have at least one output-area. They may
have multiple input-areas. These areas can be defined rather freely, by
modifying the building. For consumption the \hyperref[ch:Goods]{goods} are
placed in special buildings. \Glspl{Tool}, \gls{Equipment}, \glspl{Weapon} and
\gls{Armor} are usually transported to special storages, while \gls{Food} or
\gls{Drink} are either picked up by the
\hyperref[ch:World:Inhabitants:Sapients]{consumers} directly or at specialized
buildings.

\section{Logistic}\label{ch:Economy:Production}

The central part of the local economic network is a storage building. It is the
place workers will search for their resources, before searching in the
surrounding area. All surplus that can not be stored at the production-sites is
also transported to the next storage by the workers\footnote{ Not that workers
	only deliver surplus, once the output-area is completely full. }. To facilitate
smoother production, porters can be employed to either support the
production-sites, by carrying \hyperref[ch:Goods]{goods} to the production-site
or from the production-site to the storage, once the output-area is
half-filled\footnote{ Note that fetching goods is prioritized over delivering
	surplus, if the input areas are more than half-empty. }.

If workers or porters fail to find the desired \hyperref[ch:Goods]{goods} at a
storage building, they will raise demand for this \hyperref[ch:Goods]{good} in
the storage building. The porters assigned to the storage-building will attempt
to collect \hyperref[ch:Goods]{goods} that are in high demand first.

Demand is also generated if the amount of \hyperref[ch:Goods]{goods} stored in
a storage falls below a configurable level. Once the amount of
\hyperref[ch:Goods]{goods} exceeds a configurable level a surplus
occurs\footnote{ The surplus level is three-quarter of the total possible
	storage capacity for each \hyperref[ch;ch:Goods]{good}, by default. }. If a
storage building nearby has demand for a surplus in another storage building,
the porters will carry the surplus to the storage with the demand\footnote{
	This affect both storages, so the porters of the surplus-storage will deliver
	their surplus to the demand-storage, while the porters of the demand-storage
	will fetch the demanded resources from the surplus storage. Since porters will
	prioritize fetching demand over delivering surplus, most
	\hyperref[ch:Goods]{goods} will fetched instead of delivered.}.

Since porters only transport few resources between neighboring storages, long
distance logistics is accomplished by traders. The most common one are domestic
traders. The best way to think of them is long-way porters. These travel
between two friendly storages, usually owned by the same ruler. Like porters
they carry resources according to supply and demand. A domestic trader may
transport \gls{Food} and \gls{Drink} to a military outpost for example. A
domestic trader can deliver to any storage according to the demands of this
storage, but only collect from their own storages.

To collect from foreign storages a trade-route has to be established. This
begins by the posting of trade-offers. Once foreign trade for a
\hyperref[ch:Goods]{good} has been permitted an exchange value can be set for
it. If none is set the strength of the demand or size of the surplus determines
the exchange value. A foreign trader can now supply demanded
\hyperref[ch:Goods]{goods} and receive surplus goods in exchange. The amount of
goods bought by the trader is rounded down in favor of the storage if no coins
are available to cover the difference\footnote{ If coins are available the
	storage will prefer to pay-out the trader, before accepting coin. }. Together
with possible competition some trades may be less favorable on the arrival of
the foreign trader than on assumed on his departure\footnote{ The opposite may
	also be true.} It is therefore a possible strategy to place a trade
domestically supplied storage close to a foreign storage to ensure favorable
prices.

Alternatively exchanges rates can be fixed between two storages by two rules by
signing a a trade-contract. This is offered at one the storage and fixes the
exchange-value of a \hyperref[ch:Goods]{good} for a time if a minimal amount of
resources is delivered. These can be used to ensure a steady supply of
resources or facilitate cooperation by offering favorable prices them to
allies.

The last traders are independent traders. These hail from all
\hyperref[ch:Tribes]{tribes}, but have no allegiance to any ruler, traversing
the lands on their own. Opposed to the barter exchange of trade-routes they buy
and sell all kind of \hyperref[ch:Goods]{goods} against \glspl{Coin}\footnote{
	The rates charged by independent merchants depend on the amount of
	\hyperref[ch:Goods]{goods} they carry, but do always ensure that the trade is
	profitable to them. }.

To support porters and traders in their tasks, they are often accompanied by
\hyperref[ch:World:Inhabitants:Livestock]{livestock} carrying baskets or
wagons. For more dangerous routes traders are often combined into caravans
headed by a head-trader and accompanied by warriors\footnote{ Especially
	independent traders have very powerful escorts to protect against raids and
	\hyperref[ch:World:Inhabitants:Animals]{wild animals}. }.
