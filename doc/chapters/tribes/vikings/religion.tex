\subsection{Religion}\label{ch:Tribes:Vikings:Religion}
The \gls*{Vikings} dedicate areas to the spirits. Within theses areas the
spirits prosper and supply the \gls*{Vikings} with resources and protect these
sacred places, while they remain pristine. The sanctuaries have a inner circle
and an outer circle. One sanctuaries outer circle can not overlap with the
outer circle of another sanctuary.

All sanctuaries are employ druids. They improve the sanctuary and collect
magical resources from it. In the druid-circle great druids can be trained.
These can then use the collected magical resources to weave spells.

\subsubsection{Sanctuary of the forest}\label{ch:Tribes:Vikings:Religion:Forest}
The spirits of the forest cherish the trees and the sanctuary is strengthen by
all ancient trees in its inner and outer circle. The druids plant trees in the
inner and outer circle of the sanctuary, which produce magical resource.

\paragraph{Level 0: Sacred grove}
After the sanctuary was established the trees around it slowly turn into
ancient trees. Druids can harvest ancient bark from this trees\footnote{ Magic
	bark is just magic resource, not a special good. }. Great Druid can use the
spell \textquote{Create forest}\footnote{ A small forest appears at the target
	area of the spell. } once the sanctuary was established.

\paragraph{Level 1: Living woods}
Additional herbs and mushrooms begin to sprout in the inner and outer circle of
the sanctuary. A bear appears to defend the inner circle of the sanctuary.
Boars and deer appear in its inner circle regularly. Great druids can use the
spell \textquote{Animal kinship}\footnote{ Creates a number of boars and deer
	at the place. } once this level is reached.

\paragraph{Level 2: Ancient forest}
Ancient trees now grow chestnuts and apples. Wisps start appearing confusing
and stunning enemies walking into the inner circle of the sanctuary. The wisps
also leave gits of mushrooms and herbs at the center of the sanctuary. Great
druids can use the spell \textquote{Wisp lights}\footnote{ A few lights appear
	stunning enemy soldiers. } once this level is reached.

\paragraph{Level 3: Enchanted woods}
The sanctuary attracts a might ent. The tree-shepherd patrols the inner circle
of the sanctuary and kills enemies that dare to venture in it. He also delivers
the dead-wood of the forest as timber to its center. Thorns fill out the entire
inner circle of the forest, injuring all enemies that walk upon it. Great
druids can use the spell \textquote{The forests revenge}\footnote{ Trees within
	the area of the spell awakened. They maul close enemies and throw branches at
	them over a short distance. } once this level is reached.

\subsubsection{Sanctuary of the meadows}\label{ch:Tribes:Vikings:Religion:Grassland}
The spirits of the meadows cherish all flowers on it. The druids fell trees,
remove rocks and plant flowers within the inner and outer ring of the
sanctuary.

\paragraph{Level 0: Peaceful fields}
Once the sanctuary was established the flowers begin to bloom. From these
blooming flowers the druids can harvest the enchanted petals{ Enchanted petals
		are just magic resource, not a special good. }. Great Druid can use the spell
\textquote{Spring}\footnote{ A bunch of flowers appear. Since this is the only
	way besides the sanctuary to create flowers it is a convenient way to either
	level it up faster or get mead production going. } once the sanctuary was
established.

\paragraph{Level 1: Flower fields}
A wild bull appears defending the inner circle of the sanctuary. The grass
grows lusher and sheep and oxen grow faster on the meadow. Great druids can use
the spell \textquote{The great shedding}\footnote{ All sheep in the area of the
	spell immediately produce two units of wool. } once this level is reached.

\paragraph{Level 2: Fairy circle}
A bunch of fairies fly between the flowers, in the inner circle. They can turn
enemies into sheep for a short duration. They also assist the bees collecting
honey, while shearing the sheep delivering honey and wool to the center of the
sanctuary. Great druids can use the spell \textquote{Bee hive}\footnote{ A
	swarm of bee appears and chases enemies in a random direction, while doing
	minor damage to them. } once this level is reached.

\paragraph{Level 3: Unicorn pasture}
A herd of unicorns moves into the inner circle of the sanctuary. The will
trample and pierce unwelcome intruders with their horns. They shed their horns
on occasion leaving them in the center of the sanctuary. Once the herd grows
above a certain size unicorns will join the \gls*{Vikings} as mount to defend
the spirits. Great druids can use the spell \textquote{Unicorn
	stampede}\footnote{ Calls a herd of unicorns from the spirit realm, that
	trample from the position of the great druid in the direction of the spell.
	Trampling all in their way. } once this level is reached.

\subsubsection{Sanctuary of the sea}\label{ch:Tribes:Vikings:Religion:Sea}
This has to be built within a lake or the sea shore. The spirits of the sea
cherish the rough waves and the cold breeze. The druids transform adjacent land
into swamp and row out to the sea in boat to converse with dolphins and
manatees.

\paragraph{Level 0: Salty rocks}
Once the sanctuary was established the water tiles around it slowly become
rough waves. Willows grow in the swamp. From willows the and rough seas the
druids can bottle wailing{ Bottled wailing is just magic resource, not a
		special good. }. Great Druid can use the spell \textquote{Rich fishing
	grounds}\footnote{ The number of fish in the water increases, } once the
sanctuary was established.

\paragraph{Level 1: Great reef}
Countless fish now live in the reef\footnote{ Fish spawn next to the sanctuary.
} and only kept in the balance by a large shark that circles it within the
inner circle. In the swamp within the inner circle a giant toad preys on the
enemies of the \gls*{Vikings}. Great Druid can use the spell \textquote{Salty
	catch}\footnote{ Used on the coast, it creates a bunch of salt and fish. } once
this level is reached.

\paragraph{Level 2: Undersea harbor}
The spirits of the sea walk on the land. Within the inner sanctuary a guard of
fish-men defends the sanctuary with their spears fashioned from (whale)-bone.
They gift the druids gold in exchange for their stories. Great Druid can use
the spell \textquote{Wet warriors}\footnote{ Used on the coast it calls a few
	fish-men, that attack nearby enemies before retreating back into the depths. }
once this level is reached.

\paragraph{Level 3: Temple of the nymph}
A nymph has moved into the sanctuary. Her beauty enchanting all that see her.
If enemies are foolish enough to attack her, she will graciously accept a few
of them as her new honor-guard. They will defend her against their former
comrades until they die of starvation. She will also gift a Mithril sword to
the \gls*{Vikings} from time to time to thank them for the protection of the
sanctuary. Great Druid can use the spell \textquote{Sirens song}\footnote{ Used
	on the coast it calls all enemies towards the shore, where they strip
	themselves of weapons and armor and start to swim towards the siren until they
	drown. } once this level is reached.

\subsubsection{Mountain sanctuary}\label{ch:Tribes:Vikings:Religion:Mountain}
This has to be built next to mountain and rocks. The spirits appreciate rock
carvings and moss. The druids carve runes on rocks or lift rocks from the
ground\footnote{ Since rocks yields stone this should be a very slow process.
}.

\paragraph{Level 0: Rock field}
Once the sanctuary was established crystals start to sprout on the carved
rocks. The light of this crystals can be harvested by the druids{ This yields
		magic resource. }. Great Druid can use the spell \textquote{The bones of the
	earth}\footnote{ A bunch of rocks appear in the target area.footnote{ These can
			be harvested for stone. } } once the sanctuary was established.

\paragraph{Level 1: Crystal grotto}
A lonely cougar moves into the inner circle of the sanctuary. It attacks all
enemies of the \gls*{Vikings}. Carved-Rocks covered in crystals, explode after a
time, yielding a small amount of stone. Great Druid can use the spell
\textquote{Riches of the mountain king}\footnote{ The resources of mines in the
	area are increased by a small amount. } once this level is reached.

\paragraph{Level 2: Troll cave}
A mighty troll moves in. With his mighty pick the troll not only keeps the
inner circle save of invaders, but also mines the occasional piece of copper
ore from the crystals depositing it its hoard in the center of the sanctuary.
Great Druid can use the spell \textquote{You shall be stone}\footnote{ Turns a
	small amount of enemies and their weapons permanently into stone/rocks. So the
	houses of the \gls*{Vikings} can be built from their enemies. } once this level
is reached.

\paragraph{Level 3: Dragon peak}
A mighty dragon takes the center of the sanctuary as its domain. It defends the
inner sanctuary with its flaming breath from invaders and permits the druids to
collect its iron scales once they shed. Great Druid can use the spell
\textquote{Dragons curse}\footnote{ Turns a small amount of enemies and their
	weapons permanently into copper-, gold- and iron-bars. } once this level is
reached.

\subsubsection{Sanctuary of the hearth fire}\label{ch:Tribes:Vikings:Religion:City}
The spirits of the hearth fires love the \gls*{Vikings}, their fields and
houses\footnote{ The sanctuary levels with wheat fields and buildings in its
	inner and outer circle. Making it different other sanctuaries. }. The druids
walk from house to house lighting candles\footnote{ So this is the only
	sanctuary that consumes resources. } to delight the spirits.

\paragraph{Level 0: Hamlet}
Once the sanctuary was established the burnt candles turn into spirit-embers,
which can be collected by the druids\footnote{ Spirit embers are magical
	resource not a special good. }. Great Druid can use the spell
\textquote{Wonderful harvest}\footnote{ A bunch of ready to harvest wheat
	appears. } once the sanctuary was established.

\paragraph{Level 1: Village}
A group of mocking spirits appear and wander the inner circle of the sanctuary.
Their mocking distracts enemies stopping them from attacking. All shops and
fields passed by the spirits gain a production boost. If the spirits pass an
empty pigsty or chicken-pen a pig or chicken appears in it. Great Druid can use
the spell \textquote{Spirits gifts}\footnote{ A bunch of clothes and bread
	appear. } once this level is reached.

\paragraph{Level 2: Town}
A ghostly night-watch appears and defends the inner circle against intruders at
night.\footnote{ The night-watch are ghostly soldiers armed with spears. }. At
dawn the night watch turns into alcoholic beverages. During the day a burning
knight walks through the inner circle of the sanctuary. Ready to defend it
against invaders. At dusk the knight turns into a few units of coal. Great
Druid can use the spell \textquote{Black rider}\footnote{ A black ghostly rider
	appears. Slaying one enemy, causing nearby enemies to flee in panic. } once
this level is reached.

\paragraph{Level 3: City}
A group of animated armor suites patrols the inner circle of the
sanctuary\footnote{ They wear metal armor, helmets and either bows or swords
	and shields. }. From time to time the armors are remade by the sanctuary
leaving their components behind for the warriors of the \gls*{Vikings}. Great
Druid can use the spell \textquote{The great bell}\footnote{ A giant bell tolls
	three times shattering walls beneath it. } once this level is reached.
