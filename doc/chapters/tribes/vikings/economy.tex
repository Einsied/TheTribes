\subsection{Economy}\label{ch:Tribes:Vikings:Economy}
The economy of the \gls*{Vikings} is deeply entwined with their
\hyperref[ch:Tribes:Vikings:Religion]{religion}. For example, one of their
\hyperref[ch:Tribes:Vikings:Religion:Mountain]{sanctuaries} permits them to
regenerate \hyperref[ch:Goods:Nature:Minerals]{minerals}. Other just improve
the harvest. So the economy is intertwined with the sanctuaries by the
resources they produce and they will be listed here.

\subsubsection{Raw resources}\label{ch:Tribes:Vikings:Economy:RawInputs}
As most tribes the \gls*{Vikings} utilize the bounty nature provides them. What
raw-resource the different sectors of their economy use will be listed here.

\begin{longtable}{ll}
	\toprule
	Sector                & Natural inputs \\
	\midrule
	\Gls{Food}-production & \Gls{Game}     \\
	\bottomrule
\end{longtable}

\subsubsection{Food}\label{ch:Tribes:Vikings:Economy:Food}
The \gls*{Vikings} hunt, gather and farm for their \gls{Food}, which leads to a
number of different \gls{Food} production chains. Here they are sorted
according to their final product.

\paragraph{Hunter}
The \hyperref[ch:Tribes:Vikings:Professions]{hunter} stalks for \gls{Game} and
slays it with his \gls{Weapon}. Afterwards he skins it for \glspl{Hide} and
butchers it for \gls{Meat}. The \gls{Meat} can directly be consumed by the
\gls{Vikings}.

% Todo finally expand the economy
