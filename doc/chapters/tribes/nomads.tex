\section{\Glsfmttext{Nomads}}\label{ch:Tribes:Nomads}

\begin{flushright}
	\emph{The mount below, the wind behind, ahead the enemy.}
\end{flushright}

Among the cruel sun of the endless steps live the \gls{Nomads} in their tepees
and yurts. All expect their holy places can be move\footnote{ This also means
	that the naturally spawned defenders of these holy places are more powerful
	than their \gls{Vikings} equivalents. }. They live in tepees, yurts and wagons.
The foragers, cattle-breeders, artisans\footnote{ The artisans make weapons and
	tools from wood and leather. }, mount-breeders, butchers, lumberjacks,
meat-driers, warriors, shamans, cloth-makers and chieftains can move their
homes and workshops with ease. Since tools, tepees, yurts and wagons are made
from leather timber and woolen-cloth the economy of the \gls{Nomads} is quite
small.

\Gls{Nomads} live of their large herds. In the sunny desert they bread camels as mount
and cattle, in the temperate plain they bread oxen and horses and in the frosty north
they bread mammoths and yaks.
They survive of the dried meat of their cattle and drink their fermented milk for
entertainment.

Their workers wear woolen and leather clothes and boots, their warriors leather
armor and helmets. They produce neither shields nor metal blades. Therefore
their warriors carry spears, lances and short bows into battle.

The only luxury goods they produce is furniture and great coats. They are a
simple people and fight for the three gods of the plains, the endless sky above
their heads, the endless ground below their galloping mounts and the sun
shining on guilt and non guilty alike.

\subsection{Religion}
\Gls{Nomads} have three gods that all require different offerings.
For this offerings the shamans can cast spells, two per temple level.
One is always a boon the other a curse.
Once a temple has enough offerings it levels up.

\subsubsection{Temple of the sky}
The god of the sky demands mounts as offerings. The wind is mighty, fickle and
destructive. It bestows its power upon those that can ride with it, enabling
them to lay their foe low.

\paragraph{Level 1: Sky altar}
A swarm of crows takes the offered mounts up into the skies. Multiple swarms of
crows defend the temple attacking all enemies that come close doing small to
medium damage. The boon of this level is \textquote{Wind in our
	backs}\footnote{ Mounted warriors ride faster for a while after being affected
	by this spell. }, its curse is \textquote{Murder of crows}\footnote{ A swarm of
	crows appears and can be controlled by the player for a time. While they can
	damage enemies they might be better suited as scouts. }.

\paragraph{Level 2: Harpy peak}
A swarm of harpies appears and harasses all enemies that come close to the
temple. They do not only damage them but also sometimes stun them with their
shrieks. The mounts are now collected by a giant bird. The boon of this level
is \textquote{Long arrows}\footnote{ Affected bowmen get a significant range
	increase. }, its curse is \textquote{Winds of change}\footnote{ A strong wind
	starts to blow destroying trees and crops below it. }.

\paragraph{Level 3: Sanctum of the winged sovereign}
A giant bird appears on the temple to feasts on the mounts. It flies off to
attack enemies in the direct vicinity of the temple.

The boon of this level is \textquote{Great hunt}\footnote{ Mounted warriors
	affected by this spell, start to fly for a short while, permitting them to
	cross seas, mountains and city walls. They land once the stand still and there
	is a place to land nearby. If the spell wears off and there is no landing place
	they keep flying. }, its curse is \textquote{Great talons}\footnote{ A giant
	bird appears and destroys whatever is under it, be it trees, city walls or
	small enemy armies. }.

\subsubsection{Temple of the earth}
The god of earth is simple as the dirt below you feet. In exchange for meet it
grants you the power of the beast. The earth is steady and fair, it will feed
your people and see to their protection.

\paragraph{Level 1: Altar of bull}
A group of wild bulls patrols the area around the temple attacking all enemies.
Wild cattle spawn next to it. The boon of this level is \textquote{Gifts of
	mother earth}\footnote{ People affected by this spell will loose hunger, }, its
curse is \textquote{Tremor}\footnote{ Affected people will be stunned by the
	shacking earth. Buildings will be lightly damaged. }.

\paragraph{Level 2: Shrine of the Minotaur}
A Minotaur guard appears to defend the temple. Wild mounts spawn next to the
temple. The boon of this level is \textquote{Fertility}\footnote{ A few of the
	affected cattle and mounts immediately procreate. }, its curse is
\textquote{Curse of the horns}\footnote{ A group of enemies is temporary turned
	into cattle, leaving their equipment on the ground. This means once the spell
	ends they enemies still disarmed. }.

\paragraph{Level 3: Hall of the horned king}
A Minotaur chieftain moves into the temple defending it and occasionally eating
close by cattle. The boon of this level is \textquote{Blessing of the
	horns}\footnote{ A group of people sheds all their armor and humanity. They are
	turned into Minotaur, giving them twice the strength and appetite for meat. The
	lack of protection they compensate with their speed and the additional power of
	their blows against the enemies. }, its curse is \textquote{Thunder on the
	plains}\footnote{ A bunch of ethereal cattle appears and trampling and injuring
	everyone in their path. }.

\subsubsection{Temple of the sun}
The sun is as benevolent as vengeful. It is prideful and demands only complete
devotion from its followers. Remember it is foolish to oppose the will of the
sun\footnote{ Only one sun temple can be built, to avoid mass worship. }.

\paragraph{Level 1: Sanctum of the sun}
The sanctuary can be worshiped by the \gls{Nomads}. This will let them carry
favor with the sun. Around the temple sun-rays blind enemies, stunning them
shortly. The boon of this level is \textquote{Healing ray}\footnote{ The sun is
	benevolent. Once it touches the wounds of your warriors they will magically
	heal. Praise the sun. }, its curse is \textquote{Exhausting heat}\footnote{ The
	sun burns down on your enemies, drenching all the sweat from their body. This
	will slow them and their mounts down significantly. }.

\paragraph{Level 2: Council of the serpents}
Some of the worshiping nomads will turn into snake-humanoids. These will defend
the temple with their cooper swords, spears and shields. Once a certain number
of snake-humanoids is reached all further converted worshipers will join the
armed forces of the \gls{Nomads} instead. The boon of this level is
\textquote{Feathered mounts}\footnote{ The sun sends down mounts from a
	forgotten time. Those giant birds are not only faster than your mounts they can
	also bite the enemies had clear off. }, its curse is \textquote{Curse of the
	scale}\footnote{ The sun shines on all equally, so these cure affects friend
	and foe alike. A few warriors hit by its rays drop their armor and weapons,
	before turning into beasts from a forgotten time. These beast start attacking
	the closest people they find. }.

\paragraph{Level 3: Temple of the blinding empress}
A T-Rex appears and patrols around the temple. It attacks all enemies of the
\gls{Nomads} that dare to venture close. The boon of this level is
\textquote{Great beasts of burden}\footnote{ The sun sends down great beasts of
	burden from a forgotten time. These can carry your tepees/yurts/wagons faster
	than normal mounts. They can also carry saddles, carrying you craftsmen
	workshops, permitting them to work while on the move. }, its curse is
\textquote{Will of the sun}\footnote{ The sun shows its face and basks the area
	in its glory. Every living creature in the center of the spell is immolated.
	All burnable things catch fire. All living things around the center are
	injured, disoriented and are severely injured. }.

\subsection{Starting conditions}
\Gls{Nomads} start with a chieftain-tent, a mobile scouting/defense tower\footnote{
	A platform for a few bowmen that can be upgraded to give more visibility
	and accompany more bowmen.
	This is their only defensive building and it is weak against ground attack.
	To weak to be a really viable defense.
	But still better than being butchered on the ground.
}, a few living tents, a cattle-breeder camp and a few mounts and cattle.

\subsection{Play-style}
\Gls{Nomads} are a horde faction in more way then one.
They have no defenses so their best defense is a good offense or a fast retreat.
Considering that they can speed up the process with which they can move
their base both is an option.

Their true strength are their mounted archers allowing them to harass their
enemies as they please. The lack of a complex economy permits them to focus
fully on plundering the resources they need. \Gls{Nomads} should be always on
the attack while growing their herds behind the lines, relocating their camp to
avoid enemy retaliation. This way they will weaken others while gaining
strength. Their true powers lies with their shamans and the sacrifice they can
make to their gods.

The ability to upgrade their tepees to yurts and then to wagons improves
productivity and mobility, while increasing the number of mounts necessary for
transport\footnote{ A tepee can be loaded onto a mount or carried by a human. A
	yurt has to be loaded onto a mount. A wagon needs to mounts. }.
