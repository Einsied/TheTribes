\section{\Gls{Egyptians}}

\begin{flushright}
	\emph{Behind high walls, build for immortal glory.}
\end{flushright}

Along the fertile plains of the river irrigated fields of reed are tended and
glorious monuments erected. Within their palaces rule the priest-kings of the
\gls{Egyptians} over their people, protecting them in this life and the next.

The \gls{Egyptians} build great cities from nothing stone. Forgoing wood,
bricks and other materials their buildings store the cold of the night during
the scorching days in the desert, while being immune to fire. The only
buildings that they need to maintain are their monuments, which they furbish
with rich fabrics and paint.

Should the maintenance of the monuments be neglected, their glory will diminish
and the \gls{Egyptians} will rise against their priest-king to install one that
ensure their glory lasts eternally.

Their agriculture is based on the cultivation of wheat, flax and papyrus. They
raise small cedar forests to collect timber and resin. The wheat is milled into
flour and baked into bread, which they enjoy with fish, goat cheese and milk
and beer. As a additional treat they like to snack on figs, which they raise in
special orchards.

\Gls{Egyptians} were strongly influenced by the stark contrast between
the river and the desert in their homeland,
leading to an agriculture that is strongly reliant on irrigation.
While they may farm of simple fertile soil their yields can be drastically
increased by irrigation.
To facilitate the irrigation they build \textquote{Shrines} next to a body of sweet water.
Within the shrine resides a singular priest and their assistant.
The assistant uses a shovel to dig the ditches and maintain them.
The priest supervises the work and prays for the blessing of the local river.

Besides their local river the \gls{Egyptians} pray only two three other gods:

\begin{enumerate}
	\item The god of family and hearth fire, which they worship within their houses in
	      private. If they can obtain them by trade, they born candles in his honor. He
	      sometime reciprocates by small gifts of gold.
	\item The goddess of war. She is worshiped on the battlefield and her temple also
	      serves as mustering ground for new potential warriors. The priests bless all
	      unarmored soldiers, so they fight with greater zeal.
	\item The king of the afterlife, who was slain, cut to pieces and strewn across the
	      world. His priests are the priests of the mortuary cult. They work tirelessly
	      to prepare for his resurrection. To this end they practice the preservation and
	      reanimation of the dead. To ensure not a single soul is lost, they equip all
	      soldiers of the \gls{Egyptians} with magical talismans that teleport their
	      armor, weapons, shields and corpses back to the closest mortuary temple. There
	      they are stored until they can be wrapped in linen and anointed with resin,
	      before they are stored in a necropolis nearby.
\end{enumerate}

Their craftsmen weave flax into linen and sow linen into linen clothes. The
clothes are worn for protection against the environment and can be colored to
increase their value and the satisfaction of the wearer. The paint for this
fine linen clothes is made from resin and figs. The papyrus is crafted into
paper, on which a poet using the aforementioned paint writes beautiful poems.
These are traditionally burned after being read to ensure the experience
remains unique. These public readings in the theater are attended by many of
the \gls{Egyptians} and satisfy their desire for beauty. The only other way are
masterful wall-paintings, but these loose their appeal after a time, so the
painter needs to paint a new scene in the house after a while.

The carpenter cuts down the large cedar logs and turns them into furniture.
Since the houses of the \gls{Egyptians} are no place for heavy wooden
furniture, the carpenter uses raisin and linen to build light elegant cots and
chairs. The leftover timber is the usually sold on small river-galleys built
from it. The sails of this galleys are obviously made from timber.

To build stronger war galleys a ram is added and often reinforce with a metal
tip. The \gls{Egyptians} only mine gold, copper and coal themselves. From the
gold the goldsmith forges dead-masks and golden-jewelry. The copper is forged
into tools and swords and spears. They also use short bows and medium wooden
shields. Their hats are protected by either linen caps or copper helmets. Their
warriors also wear linen armor\footnote{ This is equivalent to leather armor. }
or light metal armor. They also breed mounts to carry the chariots, their
chariot maker furnishes from copper, wood and linen. These chariots need to be
manned by an unarmed driver and can carry up to two soldiers into battle.

In general \gls{Egyptians} prefer to build and farm instead of fighting. So
they raise impressive walls and towers from stone. While they might be rather
small in the beginning they keep rising\footnote{ They can upgrade their walls
	continually, without making them unusable. } until they are only rivaled by the
\gls{Romans}. From there their archers rain death upon their enemies until the
army of \gls{Egyptians} is strong enough to beat their enemies on an open
field.

\subsection{Monuments}
The monuments are testament to the right to rule of the local priest-king or
-queen. Their glory is the legitimacy of the current ruler. To ensure they
remain glorious monuments instead of becoming abandoned ruins they have to be
constantly refurbished. The resources necessary for this are usually linen,
furniture and paint\footnote{ If not otherwise sated all resources have to be
	supplied to maintain the glory. So having a ton of linen and no furniture may
	lead to a glory issue. }. Should the glory be insufficient for the number of
subjects currently ruled they will riot.

\subsubsection{Painted wall}
This is a wall that is regularly painted with new scenes glorifying the current
monarch. Since the stone wall was built to last for aeons, only paint is
necessary to maintain this monument. Calling it a monument is kind of a stretch
however, motivational billboard might be more fitting.

\subsubsection{Public house}
A simple luxury. A house filled with fine furniture and splendid wall
paintings\footnote{ It is not necessary to resupply color and furniture. With a
	full supply of either half the glory can be achieved. For the full amount of
	glory both are necessary. }. The constant change of furniture and scenes shows
the prosperity of this small realm and invites the subjects to relax.

\subsubsection{Clothed statue}
This statue is painted and clothed in linen. It is a over-sized representation
of the local ruler. To show the glory of the ruler the face paint and clothes
have to be always following the newest style, so they are constantly changed.

\subsubsection{Festival Plaza}
This is a splendid building decorated with gold and exquisite furniture. Within
it \gls{Egyptians} celebrate the glorious rule of their current monarch. To do
this in the appropriate fashion they require food, dates and copious amounts of
beer\footnote{ The only necessary resource is beer. All other can be supplied
	independently. Every resource increases the glory by 20\% supplying all gives
	another 20\%. }. Since the celebrations can become quite rowdy the furniture
also needs replacement from time to time.

\subsubsection{Great veiled theater}
This place is one of exotic entertainment. Every play is held behind a
transparent current of linen illuminated with coals from behind. All the actors
are only visible as shadows and the play is entirely silent. During the play
the spectators try to guess the identity of the actors and which role they
play.

After the play has finished the plot is read out aloud to satisfy the curiosity
of the spectators. Afterwards the linen curtain and the pages the play was
written on are burnt to ensure the next play will hold a new mystery.

\subsubsection{Palace}
This is the seat of a priestly monarch. To maintain the glory of this palace
its walls needs to be repainted, the linen curtains replaced and the furniture
updated to the newest tastes\footnote{ Every resource (paint, linen, furniture)
	gives 25\% of the total glory value. Supplying all 3 gives another 25\% }.

The palace differs from other monuments, because it also doubles as a defensive
structure. Its inner compound is flanked by small towers that allow archers to
defend it. This makes it an interesting choice in a forward position.

\subsubsection{Great Library}
The great library contains countless works of art and long forgotten knowledge.
On fine chairs the scholars lounge and amuse themselves with the newest
creations of the poets. Curiously the best poems always seem to go missing, so
that a constant stream of replacements has to be provided\footnote{ The
	building consumes only papyrus. }.

\subsubsection{River Baths}
The bath is built on the shores of the river. Coal is used to heat the sauna
and the pools and fresh linen-towels provided to every guest. A luxury that
shows the glory of the \gls{Egyptians}.

\subsubsection{Halls of contemplation}
This hall is filled with elegant furniture stone statues and countless
meditation candles\footnote{ Consumes furniture and candles. This implies the
	\gls{Egyptians} are trading with the \gls{Vikings}. }. The pristine calm that
fills its visitor is a feeling that no one else achieves. Together with the
sublime architecture on the inside the splendor is truly glorious.

\subsubsection{Relaxation chambers}
On the divas woolen cloth and soft pillows invite the visitor to contemplate
the sublime paintings on the wall\footnote{ The building consumes paint,
	pillows and woolen cloth. This implies the \gls{Egyptians} are trading with the
	\gls{Vikings} and the \gls{Nomads}. }. A tribute to the far reaching trade
connections of the \gls{Egyptians}.

\subsubsection{Hall of foreign delights}
In these halls food and drink from all over the world are presented\footnote{
	All non-domestic foods and drinks contribute. There is a small bonus for every
	additional, that slowly grows up to 25\% of the total glory. }. Demonstrating
the prosperous trade-routes of the \gls{Egyptians}.

\subsubsection{Intercultural exhibition}
Goods from all over the world are exhibited here\footnote{ All non-domestic
	foods and drinks contribute. There is a small bonus for every additional, that
	slowly grows up to 25\% of the total glory. }. Unfortunately they keep
disappearing, so replacements are necessary all the time.

\subsubsection{Tomb}
This is the tomb of a member of the royal court. The owner needs to be needs to
be comfortable by changing the outer layer of linen around him every day.

\subsubsection{Great Tomb}
In this tomb lays an important member of the royal court. To ensure his comfort
in the after live the outer layer of his linen wrapping is changed daily and
new furniture is burnt in front of a small altar.

\subsubsection{Royal Tomb}
A member of the royal family is buried here. In addition to the usual linen
wrappings and furniture offerings they were a number of golden masks. These
have to be reforged from time to time\footnote{ The monument turns golden mask
	into jewelry. }.

\subsubsection{Pyramid}
The pyramid is the greatest monument. Is size alone is glorious\footnote{ A
	quarter of the glory stays even if there is no maintenance. }. Within the
pyramid lies an important member of the royal family. So in addition to linen,
furniture, and golden masks the family insists of offer fine linen cloth,
dates, bread and beer in equal measures, to ensure the deceased has a decent
existence in the afterlife.

\subsection{Religion}
As can be quickly seen the mortuary cult is the most venerated and biggest of
the cults. It priest do not only embalm the dead they also maintain them within
the large necropolis, that the \gls{Egyptians} raise should they be engaged in
war. Within those complexes the mummies of the warriors lay next to their
weapons. From time to time the priests replace the outer layer of linen to
ensure the dead warrior is comfortable in the after-life. It is by this action
that the mortuary cult gains favor with the spirits of the afterlife.

These spirits grant them power over the afterlife itself. So the great priests
of the mortuary cult can exchange this favour to cast the following spells:

\subsubsection{Least death}
The most meager servant of the afterlife appears and takes the life out of all
plants in an area. Crops disappear and trees turn into coal.

\subsubsection{Lesser Death}
Death is owned a debt an it is collected int he lives of all farm animals in an
area. They wither away and become dust. Punishing their owners for opposing the
\gls{Egyptians}.

\subsubsection{Great Death}
The doors to the afterlife open and swallow people next to them. Their bodies
wither and decay as their souls are sucked into the empty plains of the former
realm of the \gls{Egyptians}s greatest god. Only their equipment, weapons,
tools, armor, clothes remain. A strong warning for all that wish to oppose
death.

\subsubsection{From aeons past}
The lands have seen countless battles and the spirit lingers. With this spell
souls are called from the afterlife and their bones reform from the ground.
With their bone-spears and shields they fight for the \gls{Egyptians} before
they and their equipment return to the dust they were made from.

\subsubsection{You shall rise}
Enemies fallen in the affected area are facing a stark choice. Fighting for the
\gls{Egyptians} for a few moments after their death or waiting eternity at the
gates of the afterlife. For most it is a easy choice.

\subsubsection{Your duty is not yet done}
Instead of teleporting back to a necropolis the affected soldiers resurrect
directly on the battlefield. The process drains most life from their bodies. so
they mummify immediately. Beware however that the spark of live that would be
necessary to return their bodies back to the temples of the mortuary cult has
left them. So these warriors will fall and lay were they are slain again. Food
for carrions birds. Desperate times require desperate measures.

\subsubsection{His weeping widow}
The king of the afterlife left behind his bereaved wive. As the gates to her
palace are torn open, her tears fall on the land infesting it with a deep
sadness. Nothing wants to live were the tears have fallen and the area turns
into desert.

\subsubsection{His vengeful son}
The king of the afterlife has a prideful son, who wishes to avenge his father.
Driven to the brink of madness by his grief he kills all that are close.
Opening the gates to his palace allows him to sally forth in his chariot and
slay all that oppose him.

\subsubsection{His singing daughters}
After their father died, their mother became inconsolable by grief ad their
brother a slave of revenge, the daughters of the king of the afterlife began to
sing hie eulogy. Opening a door to their palaces permits mortals to hear their
song. All that hear it are overcome by a great stunning grief and are unable to
move or act as they contemplate the loss of a benign god protecting their souls
after death.

\subsubsection{Vision of the past}
A door opens showing the glory of days long gone. Workers in the area work
harder.

\subsubsection{Vision of the future}
Reveals the emptiness of the afterlife to all mortals in the area. This causes
them to despair throw away their weapons and flee.

\subsubsection{Raise oh King}
Used on a tomb, great tomb or pyramid this spell creates one or multiple undead
chariots. The tomb is destroyed in the process

\subsubsection{One more time, march to glory}
Used on a necropolis this spell raises all the warriors resting within. They
march out of its gates with the weapons they were buried. This can either be
desperate last stand or a final glorious victory, because the spark of live
that would be necessary to return their bodies back to the temples of the
mortuary cult have long left them. So these warriors will fall and lay were
they are slain again. Food for carrions birds as they were always meant to be.

\subsection{Starting conditions}
The \gls{Egyptians} always settle close to a source of fresh water, where they
erect a palace for their first priest-king or -queen. The palace itself is
surrounded by a wall and small towers forming a first defensive compound. In
front of it are a few houses for the kings retainers and servants.

\subsection{Play-style}
The \gls{Egyptians} are strong in the defense and can grow very strong in a
small area. So they usually erect a citadel along a river valley and keep
increasing their populations size and economic prowess until they can devastate
their enemy in great human wave tactics.

They only reason to leave the safety of the fortress is trade for foreign
luxury goods or the hunger for stone. This hunger for stone is driven by the
need for ever greater monuments. The true limit for the number of the
\gls{Egyptians} a single priest-king can rule. This either necessitates mining
outposts, trade or expeditions, to obtain the stone and metals the
\gls{Egyptians} lack in their citadels.

The mortuary cult also encourages early aggression to permit access to the
spells of the \gls{Egyptians}. The player has to balance this and a loss of
defense. Usually a bunch of lightly armored archers will be ordered to perform
a suicidal attack to fill the necropolis as soon as the weapon production
starts rolling, but these archers might be missing on the walls when the enemy
fights back.

If the \gls{Egyptians} is permitted to raise their monuments undisturbed they
will grow in strength until they finally reopen their necropolis. Than the
living and the dead march together for the greater glory of their kingdom
crushing everything in their path.
