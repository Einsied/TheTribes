\section{\Glsfmttext{Aztecs}}\label{ch:Tribes:Atztecs}

\begin{flushright}
	\emph{The sun hungers, our blood sustains it.}
\end{flushright}

In the deep jungles the \gls{Aztecs} erect their cities. The pyramids built
from giant stone blocks tower over wooden huts. In front of the palaces and
forts of the priestly aristocracy the massed of sweat and labor to feed the
priests and heir gods.

The \gls{Aztecs} grow beans, maize and cotton in chinampas on the lakes or
fertile soil within the plain. While the crops on the chinampas soak up
moisture over time, the fertile lands needs to irrigated by rain, which only
falls occasionally or by godly intervention. While the beans will grow quite
fast, cotton and maize will only grow if properly fertilized. While charcoal
made by the charcoal maker can fulfill the need for fertilization, the
\gls{Aztecs} have more efficient methods available if they keep the serpent
happy.

Their foresters plant the small saplings that will slowly grow to giant jungle
trees. Between this trees avocados, pineapples, tomatoes and pepper are grown
in special spots. At the center of this spots monuments are erected to speed
the growth. These are collected within small huts built into the giant jungle
trees. For timber their foresters saw giant branches of the trees or cut down
all the trees foreign to them. As such the \gls{Aztecs} life in harmony with
the forests surrounding their cities.

The giant trees within the forest are covered in veins. To ensure easy passage
through the forest special vein-cutters are employed. Those cut the veins with
obsidian swords (Macuahuitl) along predetermined paths. Only them, the
collectors and the foresters posses the skill to travel through the veins
unimpeded.

Once the veins are cut and the treasures of the forests are collected they are
brought towards the cities craftsmen. There within small market stalls the
delicacies are prepared. While the \gls{Aztecs} will eat all food if famine
looms, under normal circumstances they have quite particular tastes. The
simplest stands just serf a paste of ground up beans. Their taste is bland and
the bloating not becoming of the higher orders of their society, so only the
lowest peasants will accept this food, while the slaves are left no other
choice. The craftsmen will expect the paste to be enriched with either avocado
or pepper, but they prefer a stew made from maize, pepper and tomatoes. This
food will satisfy the peasantry or serve as an emergency rations for the
warriors. The warriors subside a spread of avocado and tomato served on flat
bread made from maize, which the craftsmen consider acceptable and the peasants
luxury.

To feed the priests that rule the \gls{Aztecs}, ponds are dug into the earth,
within these ponds eels are fattened with maize taken from the fields. These
eels are then either transported to the dog breeder or served with avocado
paste, pepper and tomatoes as special treat to the craftsmen, the warriors
common diet or the poorest of the priests. A luxury for the warriors and the
staple of the priests are eels with a pepper-pineapple sauce and avocado-paste
filled flat bread on the side. To truly satisfy the priests the tiny dogs are
butchered filled with slices of pineapple, eel, avocados and served on a bed of
tomatoes with a little pepper. If the dogs are not eaten they are carried
around by the priests of the \gls{Aztecs} as status-symbols.

As different the \gls{Aztecs} are in their taste for food, they are quite
united in their taste for drink. The simplest plainest drink they say is the
ale they brew from maize. The drink for very day is pineapple-wine. The
greatest luxury is spirit distilled form maize-ale, flavored with
pineapple-wine and a small infusion of pepper. But due to its rarity the spirit
is only rarely consumed.

However the craftsmen of the \gls{Aztecs} produce more than just food and
drink. From timber lumber is sawn, which is used to build the houses of the
common people or the furniture within. What is not turned into housing or
furniture is used to fire the potters kiln, where the clay from the clay-pit is
formed into beautiful ceramics which always seem to break.

While the stone-cutters primarily work to supply the ever growing pyramids and
palaces, the stones are also used by tool- and weapon-makers of the
\gls{Aztecs}. From broken rock and timber they furnish hows for the farmers,
axes for the foresters, and knives for the dog breeders.

For clothing they weave cotton into simple garments for the peasants and
craftsmen, woven armor for the warriors and robes for the priests.

The woven armor protects the \gls{Aztecs} just like a leather armor would. For
weapons they use slings made from cotton, atlatls made from timber and cotton
or Macuahuitl made from the sacred obsidian. If they wish to take prisoners
they use heavy wooden clubs. A single of their warriors might not seem to stand
much of a change against the other people, but the \gls{Aztecs} never fight
alone.

To protect their large number the \gls{Aztecs} erect large earthen ramparts,
which sides are reinforced by stone. The reason the \gls{Aztecs} have no direct
siege weapons, might be that only the mightiest of siege engines stand a chance
to grind them down, so the siege tactics of the \gls{Aztecs} focus on
overwhelming numbers. They prefer to pepper their enemies with stones from
their slings, before surging ladders to the walls and engage them in bloody
melee. For others this may seem absurd but it permits the \gls{Aztecs} to bring
their large numbers into play and take as many of the enemies prisoner as
possible.

\subsection{Religion}
Central to the \gls{Aztecs} religion is the ever-dying sun. At dusk the sun god
enters the under-world here he battles against the demons of darkness to save
the world. As the sun is mighty it defeats the demons with ease but their
attacks leave wounds. From this wounds the suns life-force oozes out as light.

So every dawn the sun rises to recover its lost vitality, while its life force
warms the surface of the world. But to mend its wounds the sun requires
sustenance, this sustenance can only be provided by the great serpents, which
it consumes once the reach maturity.

The legends of the \gls{Aztecs} say that once the sun is forgotten, the temples
abandoned and the great serpents day out, the sun will shine brighter every day
until all the world burns in its light. And after the surface evaporates the
upper and the underworld will merge and the sun will finally die, before a dark
sun is borne from the sprawling masses of the demons of darkness.

The \gls{Aztecs} have many gods, but for them only the sun rules supreme. All
other gods are just seen as helpers or tools to ensure that the sun does live
eternal. They see themselves as the only bastion against the rising of the dark
sun. Their temple show their reference for the gods they honour, but foremost
the temples are part of their always expanding efforts to mend the sun, and
bring about the age of blessed twilight, were the sun is healed so far that
after its rise it darkens as its mends with the serpents offered by the
\gls{Aztecs}.

\subsubsection{Shrine to the crying god}
Among the gods the crying god is most fond of humans and the \gls{Aztecs} in
general. He abhors their death, and whenever a human dies hie tears fall down
from the sky soaking the grounds.

It is said that ages ago he was the laughing god, a god of joy and laughter,
ruling supremer as god of the simple farmers, that were the ancestors of the
\gls{Aztecs}. But after the \gls{Aztecs} were called to nurture the ever dying
sun, he was pushed from his place and now watches in sorrow at the daily
atrocities they commit to save the world.

While he can not save the world he is still dear to most of the \gls{Aztecs}.
At his shrines young lovers pray for happiness, parents for the return of lost
children, priests for forgiveness and warriors for a painless death. All this
prayers have to be uttered in secret though, because there is only one ceremony
officially permitted. The sacrifice for rain. Up on the altar of the shrine a
human is killed in front of the crying gods eyes, so his tears flow from the
sky and water the fertile fields around, so the crops can grow.

\subsubsection{Temple cooking gods}
Food plays a central role in everyday life of the \gls{Aztecs} and its
religious aspects are governed by a twin deity. The cooking gods.

It is said that as the world was young the cooking gods tried to find
appreciation for their meals, but no spirit nor god ever liked their maize
gruel. So they decided to shape people from the gruel to and feed it to them.
If they praised the food the cooking gods sent them out into the world to
praise the height of their craft. If the people however did negatively critique
their cuisine they returned them to the gruel they were made from. The last
part of the story is mor often told to picky eaters than the first one.

Apart from smaller ceremonies focused on food and kitchen courses the large pot
is used for the great gruel. The great gruel is prepared by the priests from
water an maize. Once it starts boiling a member of the \gls{Aztecs} or a
prisoner can be drained of most of their blood. One the blood mixes with the
gruel the priests speak their prayers and from the pot a member of the
\gls{Aztecs} arises, as pale as the one drained earlier. Both wander off into
the city to take a few hearty meals, once their blood has slowly replenished
their are again just normal member of the \gls{Aztecs}. In this way the
population of a city governed by the \gls{Aztecs} can swell faster than that of
any other people, if enough food is available.

\subsubsection{Temple of the burning mountain}
The god of the burning mountain dwells in the volcano next to the great capital
city. There on its slopes the \gls{Aztecs} first encountered pepper, which for
them resembles the life-force the burning mountain absorbed from the sun. They
harvest these and sacrifice it to them, so that is fiery breath melts stone
into sacred obsidian.

This temple is hard to construct but necessary for every great city of the
\gls{Aztecs}. Without it the veins of the jungle impede their everyday
movements and their soldiers have to fight without their feared Macuahuitl.

\subsubsection{The serpent stable}
At the center of every settlements the \gls{Aztecs} found is a holy sky
serpent. This is not only often true in a geographical sense, but more so in an
economical, religious and cultural sense. All activities the \gls{Aztecs}
perform are aiming to fatten this serpent, so it can be consumed by the
ever-dying sun. Should they neglect this duty for to long the serpent may
escape its prison and feast on the \gls{Aztecs} in its proximity.

Should enemies approach the serpent will leave its pit and feast on them. It
teeth are sharp and bite even through the hardest metal, so the only limit to
the devastation it causes in the enemy ranks is the capacity of its stomach and
the speed of its teeth. In its juvenile form only a well equipped band of
raiders with either great strength in numbers or excellent ranged weaponry has
a chance of piercing its scales. So even without a single warrior the
settlements of the \gls{Aztecs} are well defended.

\paragraph{Pit}
Every settlement of the \gls{Aztecs} is founded by digging a pit. Within the
pit a single sky serpent egg is placed. Around it a wooden stockade is erected
to protect the egg from wild animals and scoundrels.

While the egg begins to hatch a small hut is built on the border of the pit.
Here lives the first priest of the sky serpent. He ensures the health and
wellbeing of the serpent and administers the human sacrifices. Every sacrifice
helps the serpent to grow mightier and stronger, while also improving its
favor.

The residue it produces after every sacrifice fertilizes the fields of the
\gls{Aztecs}, helping their crops grow faster and thereby driving the growth of
the city state.

In its juvenile form the serpent is quite weak. It scales can be penetrated by
hardened weapons thrust with sufficient force. Its magic is limited to casting
a thunderbolt from the sky next to the serpent-priest.

\paragraph{Serpent Rampart}
The pit remains the center of the hamlet while it slowly grows into a village.
To protect the serpent from their enemies the \gls{Aztecs} begin with the
construction of a wall stone and dirt wall surrounding the pit. They also
expand the simple hut into stone quarters, making enough room for two
additional priests.

Once the rampart is completed the serpent begins to feast and grow again. As
its scales harden and its fang lengthen, it becomes a more fearsome monster.
Its magical prowess begins to increase to, permitting the priests to summon
clouds with venom rain. The rain does not kill the afflicted but merely
incapacitates, os they can be taken prisoner and feed to the serpent.

\paragraph{Serpent Fort}
As the village grows into a town the hunger of the serpent grows and the
\gls{Aztecs} prepare accordingly. The simple ramparts are expanded into a fort
fitting for the warriors, the young city state raises. Within the fort the
priestly quarters are expanded to permit five priests to take care of the
serpent.

The cities warriors will raid for prisoners. Beating their enemies unconscious
with heavy wooden clubs. To support them and enable their raids to succeed the
sky-serpents teeth will rip through space and create a two-way corridor between
the fort and a scouted position.

While the serpent can open the corridor at will it can not close it. So the
\gls{Aztecs} have to be sure that they can beat the enemies on the other side
otherwise they may have doomed themselves and their serpent. Especially since
the serpents scales harden to its final strength during this stage, only
penetrable be the strongest weapons.

\paragraph{Serpent Temple}
As the town grows into a city the fort is reconstructed into a temple, the
ceiling closed, only a pair of gates at the top permits access to the serpent.
Nine priests care for it and the eggs it will lay during this stage. The eggs
permitting the expansion of the young kingdom of the \gls{Aztecs}.

With the daily sacrifices the serpent can now bestow the warriors of the
\gls{Aztecs} with a skin of scales. For a short while their warriors skin
becomes as hard as the serpents scales making them immune to most blades and
ranged weapons.

\paragraph{Serpent Pyramid}
At the last stage the temple is completed as a large pyramid. In its center a
giant serpent slowly growing the wings it needs to lift itself up into the sky.
This is the final stage of the temple and the serpent. Deep in the bowls of the
temple a few of the thirteen priests are already nurturing the serpent that
will replace the current one, once it was fed to the sun.

Once the pyramid is complete the hunger of the serpent increases further. To
satisfy it, it can create a corridor between any point its priests can see and
its maw. For a few seconds the corridor will open and all the enemies will fall
straight into its maw permitting it to consumer them.

Once the serpent has fully grown it will leave the pyramid through the gates
and fly towards the sky. During the night it will fly around the pyramid
investigating its surrounding. Once the sun rises the serpent will be devoured
by it strengthening the sun and perpetuating the cycle of life and death.

After the fully grown serpent was consumer, the juvenile at the bottom of the
pyramid will now be fed by the priests until it reaches the same size. Since
the pyramid already exists and the serpent chews quite fast, the only real
limit for the repetition of the cycle is availability of food. So the citizen
of the metropolis and the columns of prisoners will forever ascend the stairs
of the pyramid to fed the serpent so it can fed the ever-dying sun, fighting
back the darkness.

\subsubsection{Pyramid of the ever-dying sun}
The Pyramid of the ever-dying sun is a building as large as the serpent stable.
In its center is a large vessel. Within this vessel the sun-priests collect the
life force of the sun. A few drops every day slowly refilling it.

It would take a long while to fill the vessel just by the daily sun shine, but
whenever a sky serpent is consumed the sun gives enough life-force to fill the
vessel.

So the sun priest can only really work their magic once the serpents stables
have become productive. A especially wealthy metropolis might even have
multiple sun-pyramids to store all the sun-shine generated by the steady
consumption of the sky serpents.

\paragraph{Holy sunshine}
Within a large the life-force of the sun falls as rain from the sky healing the
warriors of the \gls{Aztecs} and burning their enemies.

\paragraph{Blinding Light}
The suns blinding light immobilizes all enemies carrying ranged weapons.

\paragraph{The ray of fire}
The sun sends out a great immolating beam burning through wood, stone and
enemies of the \gls{Aztecs}. No matter how hight the walls are that their
enemies built, this ray will burn them down.

\paragraph{Darkness}
The priests channel the life force of the sun back at her in the morning, so no
life-force is lost during its passage. The \gls{Aztecs} rejoice their warriors
fighting harder while all other people despair. No crops grow unless watered
and fertilized and the wild animals hide in the wilderness. While their enemies
economy crumbles the \gls{Aztecs} lay siege to their cities, beating their
demoralized enemies.

\subsubsection{Forest deities}
Within the forest the \gls{Aztecs} grow avocados, pineapples, tomatoes and
pepper. While they can be grown all over the forest the \gls{Aztecs} prefer to
grow them on special places, encouraging growth with sacrifices to minor
deities, which opposed to the major deities reside within the forest. Their
influence reaches roughly around their sacred place and is rather weak. They
are deities of growth and protection, warding of enemies and increasing the
growth of their favored plants around them.

Usually the monuments of the minor deities and the growing places are combined
into minor forest pantheons next to a gather-hut to permit more easy harvest
and worship by the \gls{Aztecs} in the forest.

\paragraph{The old root}
The old root is majorly concerned with the growth of the trees, especially the
avocado tree. It is worshipped at a great stone steel decorated with obsidian.
Around it the \gls{Aztecs} plant Avocado trees. If the grove is large enough
the old root can use them to ensnare intruders without harming them.

Since it is a deity of the forest and the veins, it despises the Macuahuitl of
the \gls{Aztecs} and the burning mountain. If Macuahuitl are brought to the
stele they are turned back into stone and the trees grow faster.

\paragraph{The sweet embrace}
The sweet embrace is a goddess of delight and deceit. She is worshipped at a
statute representing her. Below the waist she has the body of a serpent, above
the waste the body of a young women. Her head takes the form of a pineapple.
Holy to her are the pineapples and venomous snakes. The snakes can be fed with
fattened eels to gain her favor, in exchange she will sweeten pineapples more
quickly.

While the pineapples offers sweet earthly delights, between them the snakes
wait for intruders. Once bitten they will weaken, the venom slowly destroying
their health if they fail to retreat or destroy her sanctum.

\paragraph{Tomatoes}
The great bloom sees as its domain all the smaller herbs and bushes within the
forest. While the veins are the most numerous its special gift to the
\gls{Aztecs} are tomatoes, growing around its sacred mosaic. It is a cheerful
deity concerned with life and fertility as such it demands nothing more for its
favor than a token offer of a few beans to increase the growth of the tomatoes.

If enough tomatoes grow around it, it will surround nearby buildings with a
layer of protective veins. These veins are harmless to the \gls{Aztecs}, but
will throw their thorns against any intruder that attempts to destroy the
protected buildings.

\paragraph{The screaming face}
The screaming face is a protective deity, ensuring the safety of fools and
children by scaring them away from the forest. To facilitate his function the
\gls{Aztecs} erect large stone heads within the jungle, around wich grows a
large amount of pepper. To facilitate the growth of the pepper the spray the
face of the head with a small amount of their blood. Once the pepper fully
surrounds the stone head the screaming face can create from it a mist that
drives of any invaders approaching it.

\subsection{Starting conditions}
Settlement of the \gls{Aztecs} stat quite simple with the sky-serpent pit, its
resident priest a little fertile land and a bunch of peasants. These peasants
have to build huts, cut down lumber plant trees and multiply to feed the
serpent in the center of their settlement.

\subsection{Play-style}
A giant horde in a buzzing city. The complex diet and the threat of the
serpents hunger keep the \gls{Aztecs} on their toes. Eat good and be not eaten
is a central theme for them. Their people are also integrated into their
economy not only as workers but also as resource. The constant need for human
lives, makes the occasional offensive quite cheap compared to the cost of
running a city of the \gls{Aztecs}.

While their walls offer them safety, they should raid their neighbors to feed
the serpent. A strong and fully upgraded serpent pyramid will make the
\gls{Aztecs} terrible foes, which can annihilate an entire city just to further
grow in power by the very act.

The jungle may seem like a great treasure and an excellent defensive perimeter
until the enemies replace their assault forces by a humble lumberjack and watch
the \gls{Aztecs} despair as the trees they have been nurturing for half an
eternity are turned into charcoal and cheap furniture. Their enemies risk a
lumberjack the \gls{Aztecs} their entire food economy.

The \gls{Aztecs} economy even stretches to the \gls{Aztecs} themselves. People
are sacrifice to create more food and more people slowly snowballing into a
large horde. While other peoples loot cities for the resources within the
\gls{Aztecs} loot them for prisoners, that they can feed into their economy.
For them people are just another resource.
