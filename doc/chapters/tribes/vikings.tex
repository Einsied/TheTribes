\section{\Glsfmttext{Vikings}}\label{ch:Tribes:Vikings}

\begin{flushright}
	\emph{One with the spirits, one with nature.}
\end{flushright}

Deep within the woods, the blooming meadows live the \gls*{Vikings}. They have
a close connection to nature and its spirits, which they venerate in special
sanctuaries.

In the forest their hunters stalk deer and boars, among the druids and
foragers. The foragers collect herbs and mushrooms for the production of
medicine and (herb) flavored mead and mushroom ale. The wheat for the ale is
grown in small fields in front of their towns, where their craftsmen produce
golden-jewelry, drinking horns, pillows, woolen clothes, candles, leather shoes
and wooden and metal tools. To feed them the backers bake bread and delicious
chestnut- and apple pies. Should the miller be sick, the instead feast on meat,
fish and apples.

Their houses are built from timber, boards and stones. Their hamlets protected
by simple palisades, their villages by wooden walls and towers and their cities
by stone walls, towers and the spirits the so cherish.

While the fish are caught on the shore, sheep and oxen are raised on the
meadows and chicken and pigs fed with wheat and chestnuts. Behind the pastures,
wheat-fields and orchards you can find the mountains. If they do not turn them
into sacred hallowed places, they mine for copper, iron, gold and coal. They
use this metals to make various tool. The most beloved of this tools is the ax,
used by their lumberjacks, the butchers and the warriors alike.

If the wood-burner works diligently and enough metal is found in the mountain,
their weapon smith will forge battle-axes, spears and swords. Luckily he also
works as a bow-maker on the side supplying them with short bows.

Tho protect their warriors they manufacture medium wooden and medium metal
shields, leather and light metal armor. To protect the heads of their warriors
they use leather caps, helmets and horned helmets. The latter one are seen as
quite prestigious among them and are given to the most important warriors to
boost their morale in battle.

\chapter{Economy}\label{ch:Economy}

To thrive the \hyperref[ch:Tribes]{tribes} need to produce different
\hyperref[ch:Good]{goods}. Their production can be roughly divided into the
gathering of \hyperref[ch:Goods:Nature]{raw inputs}, their refinement and use
or consumption. This means \hyperref[ch:Goods]{goods} are first gathered then
transported along a production-chain, before they reach a final destination for
use or consumption.

\section{Production}\label{ch:Economy:Production}

Production occurs at production-sites. For resource gathering operations these
are the centers of work. So a \glslink{CopperOre}{copper}-miner will grab their
\gls{Pick} from the mine and walk to the next resource node containing
\gls{CopperOre}. After mining the ore he will then deliver the \gls{CopperOre}
to the mine, were it is stored in the mines output area.

If the \gls{CopperOre} has to be transformed into a \gls{Tool} it is
transported to the \gls{CopperOre}-input of a smithy. There the smith
transforms it into a \gls{Tool} using \gls{Coal} from the \gls{Coal}-input.
Then the smith places it in the output-area of the smithy.

This means that all production-sites have at least one output-area. They may
have multiple input-areas. These areas can be defined rather freely, by
modifying the building. For consumption the \hyperref[ch:Goods]{goods} are
placed in special buildings. \Glspl{Tool}, \gls{Equipment}, \glspl{Weapon} and
\gls{Armor} are usually transported to special storages, while \gls{Food} or
\gls{Drink} are either picked up by the
\hyperref[ch:World:Inhabitants:Sapients]{consumers} directly or at specialized
buildings.

\section{Logistic}\label{ch:Economy:Production}

The central part of the local economic network is a storage building. It is the
place workers will search for their resources, before searching in the
surrounding area. All surplus that can not be stored at the production-sites is
also transported to the next storage by the workers\footnote{ Not that workers
	only deliver surplus, once the output-area is completely full. }. To facilitate
smoother production, porters can be employed to either support the
production-sites, by carrying \hyperref[ch:Goods]{goods} to the production-site
or from the production-site to the storage, once the output-area is
half-filled\footnote{ Note that fetching goods is prioritized over delivering
	surplus, if the input areas are more than half-empty. }.

If workers or porters fail to find the desired \hyperref[ch:Goods]{goods} at a
storage building, they will raise demand for this \hyperref[ch:Goods]{good} in
the storage building. The porters assigned to the storage-building will attempt
to collect \hyperref[ch:Goods]{goods} that are in high demand first.

Demand is also generated if the amount of \hyperref[ch:Goods]{goods} stored in
a storage falls below a configurable level. Once the amount of
\hyperref[ch:Goods]{goods} exceeds a configurable level a surplus
occurs\footnote{ The surplus level is three-quarter of the total possible
	storage capacity for each \hyperref[ch;ch:Goods]{good}, by default. }. If a
storage building nearby has demand for a surplus in another storage building,
the porters will carry the surplus to the storage with the demand\footnote{
	This affect both storages, so the porters of the surplus-storage will deliver
	their surplus to the demand-storage, while the porters of the demand-storage
	will fetch the demanded resources from the surplus storage. Since porters will
	prioritize fetching demand over delivering surplus, most
	\hyperref[ch:Goods]{goods} will fetched instead of delivered.}.

Since porters only transport few resources between neighboring storages, long
distance logistics is accomplished by traders. The most common one are domestic
traders. The best way to think of them is long-way porters. These travel
between two friendly storages, usually owned by the same ruler. Like porters
they carry resources according to supply and demand. A domestic trader may
transport \gls{Food} and \gls{Drink} to a military outpost for example. A
domestic trader can deliver to any storage according to the demands of this
storage, but only collect from their own storages.

To collect from foreign storages a trade-route has to be established. This
begins by the posting of trade-offers. Once foreign trade for a
\hyperref[ch:Goods]{good} has been permitted an exchange value can be set for
it. If none is set the strength of the demand or size of the surplus determines
the exchange value. A foreign trader can now supply demanded
\hyperref[ch:Goods]{goods} and receive surplus goods in exchange. The amount of
goods bought by the trader is rounded down in favor of the storage if no coins
are available to cover the difference\footnote{ If coins are available the
	storage will prefer to pay-out the trader, before accepting coin. }. Together
with possible competition some trades may be less favorable on the arrival of
the foreign trader than on assumed on his departure\footnote{ The opposite may
	also be true.} It is therefore a possible strategy to place a trade
domestically supplied storage close to a foreign storage to ensure favorable
prices.

Alternatively exchanges rates can be fixed between two storages by two rules by
signing a a trade-contract. This is offered at one the storage and fixes the
exchange-value of a \hyperref[ch:Goods]{good} for a time if a minimal amount of
resources is delivered. These can be used to ensure a steady supply of
resources or facilitate cooperation by offering favorable prices them to
allies.

The last traders are independent traders. These hail from all
\hyperref[ch:Tribes]{tribes}, but have no allegiance to any ruler, traversing
the lands on their own. Opposed to the barter exchange of trade-routes they buy
and sell all kind of \hyperref[ch:Goods]{goods} against \glspl{Coin}\footnote{
	The rates charged by independent merchants depend on the amount of
	\hyperref[ch:Goods]{goods} they carry, but do always ensure that the trade is
	profitable to them. }.

To support porters and traders in their tasks, they are often accompanied by
\hyperref[ch:World:Inhabitants:Livestock]{livestock} carrying baskets or
wagons. For more dangerous routes traders are often combined into caravans
headed by a head-trader and accompanied by warriors\footnote{ Especially
	independent traders have very powerful escorts to protect against raids and
	\hyperref[ch:World:Inhabitants:Animals]{wild animals}. }.


\subsection{Religion}\label{ch:Tribes:Vikings:Religion}
The \gls{Vikings} dedicate areas to the spirits. Within theses areas the
spirits prosper and supply the \gls{Vikings} with resources and protect these
sacred places, while they remain pristine. The sanctuaries have a inner circle
and an outer circle. One sanctuaries outer circle can not overlap with the
outer circle of another sanctuary.

All sanctuaries are employ druids. They improve the sanctuary and collect
magical resources from it. In the druid-circle great druids can be trained.
These can then use the collected magical resources to weave spells.

\subsubsection{Sanctuary of the forest}\label{ch:Tribes:Vikings:Religion:Forest}
The spirits of the forest cherish the trees and the sanctuary is strengthen by
all ancient trees in its inner and outer circle. The druids plant trees in the
inner and outer circle of the sanctuary, which produce magical resource.

\paragraph{Level 0: Sacred grove}
After the sanctuary was established the trees around it slowly turn into
ancient trees. Druids can harvest ancient bark from this trees\footnote{ Magic
	bark is just magic resource, not a special good. }. Great Druid can use the
spell \textquote{Create forest}\footnote{ A small forest appears at the target
	area of the spell. } once the sanctuary was established.

\paragraph{Level 1: Living woods}
Additional herbs and mushrooms begin to sprout in the inner and outer circle of
the sanctuary. A bear appears to defend the inner circle of the sanctuary.
Boars and deer appear in its inner circle regularly. Great druids can use the
spell \textquote{Animal kinship}\footnote{ Creates a number of boars and deer
	at the place. } once this level is reached.

\paragraph{Level 2: Ancient forest}
Ancient trees now grow chestnuts and apples. Wisps start appearing confusing
and stunning enemies walking into the inner circle of the sanctuary. The wisps
also leave gits of mushrooms and herbs at the center of the sanctuary. Great
druids can use the spell \textquote{Wisp lights}\footnote{ A few lights appear
	stunning enemy soldiers. } once this level is reached.

\paragraph{Level 3: Enchanted woods}
The sanctuary attracts a might ent. The tree-shepherd patrols the inner circle
of the sanctuary and kills enemies that dare to venture in it. He also delivers
the dead-wood of the forest as timber to its center. Thorns fill out the entire
inner circle of the forest, injuring all enemies that walk upon it. Great
druids can use the spell \textquote{The forests revenge}\footnote{ Trees within
	the area of the spell awakened. They maul close enemies and throw branches at
	them over a short distance. } once this level is reached.

\subsubsection{Sanctuary of the meadows}\label{ch:Tribes:Vikings:Religion:Grassland}
The spirits of the meadows cherish all flowers on it. The druids fell trees,
remove rocks and plant flowers within the inner and outer ring of the
sanctuary.

\paragraph{Level 0: Peaceful fields}
Once the sanctuary was established the flowers begin to bloom. From these
blooming flowers the druids can harvest the enchanted petals{ Enchanted petals
		are just magic resource, not a special good. }. Great Druid can use the spell
\textquote{Spring}\footnote{ A bunch of flowers appear. Since this is the only
	way besides the sanctuary to create flowers it is a convenient way to either
	level it up faster or get mead production going. } once the sanctuary was
established.

\paragraph{Level 1: Flower fields}
A wild bull appears defending the inner circle of the sanctuary. The grass
grows lusher and sheep and oxen grow faster on the meadow. Great druids can use
the spell \textquote{The great shedding}\footnote{ All sheep in the area of the
	spell immediately produce two units of wool. } once this level is reached.

\paragraph{Level 2: Fairy circle}
A bunch of fairies fly between the flowers, in the inner circle. They can turn
enemies into sheep for a short duration. They also assist the bees collecting
honey, while shearing the sheep delivering honey and wool to the center of the
sanctuary. Great druids can use the spell \textquote{Bee hive}\footnote{ A
	swarm of bee appears and chases enemies in a random direction, while doing
	minor damage to them. } once this level is reached.

\paragraph{Level 3: Unicorn pasture}
A herd of unicorns moves into the inner circle of the sanctuary. The will
trample and pierce unwelcome intruders with their horns. They shed their horns
on occasion leaving them in the center of the sanctuary. Once the herd grows
above a certain size unicorns will join the \gls{Vikings} as mount to defend
the spirits. Great druids can use the spell \textquote{Unicorn
	stampede}\footnote{ Calls a herd of unicorns from the spirit realm, that
	trample from the position of the great druid in the direction of the spell.
	Trampling all in their way. } once this level is reached.

\subsubsection{Sanctuary of the sea}\label{ch:Tribes:Vikings:Religion:Sea}
This has to be built within a lake or the sea shore. The spirits of the sea
cherish the rough waves and the cold breeze. The druids transform adjacent land
into swamp and row out to the sea in boat to converse with dolphins and
manatees.

\paragraph{Level 0: Salty rocks}
Once the sanctuary was established the water tiles around it slowly become
rough waves. Willows grow in the swamp. From willows the and rough seas the
druids can bottle wailing{ Bottled wailing is just magic resource, not a
		special good. }. Great Druid can use the spell \textquote{Rich fishing
	grounds}\footnote{ The number of fish in the water increases, } once the
sanctuary was established.

\paragraph{Level 1: Great reef}
Countless fish now live in the reef\footnote{ Fish spawn next to the sanctuary.
} and only kept in the balance by a large shark that circles it within the
inner circle. In the swamp within the inner circle a giant toad preys on the
enemies of the \gls{Vikings}. Great Druid can use the spell \textquote{Salty
	catch}\footnote{ Used on the coast, it creates a bunch of salt and fish. } once
this level is reached.

\paragraph{Level 2: Undersea harbor}
The spirits of the sea walk on the land. Within the inner sanctuary a guard of
fish-men defends the sanctuary with their spears fashioned from (whale)-bone.
They gift the druids gold in exchange for their stories. Great Druid can use
the spell \textquote{Wet warriors}\footnote{ Used on the coast it calls a few
	fish-men, that attack nearby enemies before retreating back into the depths. }
once this level is reached.

\paragraph{Level 3: Temple of the nymph}
A nymph has moved into the sanctuary. Her beauty enchanting all that see her.
If enemies are foolish enough to attack her, she will graciously accept a few
of them as her new honor-guard. They will defend her against their former
comrades until they die of starvation. She will also gift a Mithril sword to
the \gls{Vikings} from time to time to thank them for the protection of the
sanctuary. Great Druid can use the spell \textquote{Sirens song}\footnote{ Used
	on the coast it calls all enemies towards the shore, where they strip
	themselves of weapons and armor and start to swim towards the siren until they
	drown. } once this level is reached.

\subsubsection{Mountain sanctuary}\label{ch:Tribes:Vikings:Religion:Mountain}
This has to be built next to mountain and rocks. The spirits appreciate rock
carvings and moss. The druids carve runes on rocks or lift rocks from the
ground\footnote{ Since rocks yields stone this should be a very slow process.
}.

\paragraph{Level 0: Rock field}
Once the sanctuary was established crystals start to sprout on the carved
rocks. The light of this crystals can be harvested by the druids{ This yields
		magic resource. }. Great Druid can use the spell \textquote{The bones of the
	earth}\footnote{ A bunch of rocks appear in the target area.footnote{ These can
			be harvested for stone. } } once the sanctuary was established.

\paragraph{Level 1: Crystal grotto}
A lonely cougar moves into the inner circle of the sanctuary. It attacks all
enemies of the \gls{Vikings}. Carved-Rocks covered in crystals, explode after a
time, yielding a small amount of stone. Great Druid can use the spell
\textquote{Riches of the mountain king}\footnote{ The resources of mines in the
	area are increased by a small amount. } once this level is reached.

\paragraph{Level 2: Troll cave}
A mighty troll moves in. With his mighty pick the troll not only keeps the
inner circle save of invaders, but also mines the occasional piece of copper
ore from the crystals depositing it its hoard in the center of the sanctuary.
Great Druid can use the spell \textquote{You shall be stone}\footnote{ Turns a
	small amount of enemies and their weapons permanently into stone/rocks. So the
	houses of the \gls{Vikings} can be built from their enemies. } once this level
is reached.

\paragraph{Level 3: Dragon peak}
A mighty dragon takes the center of the sanctuary as its domain. It defends the
inner sanctuary with its flaming breath from invaders and permits the druids to
collect its iron scales once they shed. Great Druid can use the spell
\textquote{Dragons curse}\footnote{ Turns a small amount of enemies and their
	weapons permanently into copper-, gold- and iron-bars. } once this level is
reached.

\subsubsection{Sanctuary of the hearth fire}\label{ch:Tribes:Vikings:Religion:City}
The spirits of the hearth fires love the \gls{Vikings}, their fields and
houses\footnote{ The sanctuary levels with wheat fields and buildings in its
	inner and outer circle. Making it different other sanctuaries. }. The druids
walk from house to house lighting candles\footnote{ So this is the only
	sanctuary that consumes resources. } to delight the spirits.

\paragraph{Level 0: Hamlet}
Once the sanctuary was established the burnt candles turn into spirit-embers,
which can be collected by the druids\footnote{ Spirit embers are magical
	resource not a special good. }. Great Druid can use the spell
\textquote{Wonderful harvest}\footnote{ A bunch of ready to harvest wheat
	appears. } once the sanctuary was established.

\paragraph{Level 1: Village}
A group of mocking spirits appear and wander the inner circle of the sanctuary.
Their mocking distracts enemies stopping them from attacking. All shops and
fields passed by the spirits gain a production boost. If the spirits pass an
empty pigsty or chicken-pen a pig or chicken appears in it. Great Druid can use
the spell \textquote{Spirits gifts}\footnote{ A bunch of clothes and bread
	appear. } once this level is reached.

\paragraph{Level 2: Town}
A ghostly night-watch appears and defends the inner circle against intruders at
night.\footnote{ The night-watch are ghostly soldiers armed with spears. }. At
dawn the night watch turns into alcoholic beverages. During the day a burning
knight walks through the inner circle of the sanctuary. Ready to defend it
against invaders. At dusk the knight turns into a few units of coal. Great
Druid can use the spell \textquote{Black rider}\footnote{ A black ghostly rider
	appears. Slaying one enemy, causing nearby enemies to flee in panic. } once
this level is reached.

\paragraph{Level 3: City}
A group of animated armor suites patrols the inner circle of the
sanctuary\footnote{ They wear metal armor, helmets and either bows or swords
	and shields. }. From time to time the armors are remade by the sanctuary
leaving their components behind for the warriors of the \gls{Vikings}. Great
Druid can use the spell \textquote{The great bell}\footnote{ A giant bell tolls
	three times shattering walls beneath it. } once this level is reached.


\subsection{Starting conditions}\label{ch:Tribes:Vikings:Start}
The \gls*{Vikings} start with a great wooden hall. This is the center of their
community, where they can store resources and train warriors. The balconies of
the hall can also be manned by bowmen and the doors locked Turning this
building in a defensive point.

\subsection{Play-style}\label{ch:Tribes:Vikings:PlayStyle}
They are rather generic, growing food, mining for ore, felling trees. Their
gimmick are the sanctuaries, forcing them to carefully consider the use of
their territory. The sanctuaries give them a great natural defense and infinite
resources for little work. On the offensive they can rely on their medium
warriors, but are better of to use the spells of their great druids. In the
early game they can use their tool-smithy to forge axes and start raiding their
neighbors, before those build a working weapon production line. For
siege-warfare they rely on ladders and \textquote{The great bell}-spell. With
their ability to build longboats and upgrade their buildings from simple tents,
to huts, to wooden houses and then to stone houses, they are quite static. They
usually defend themselves with palisades, wooden walls and towers and in the
late game with stone walls.

Their true weakness is the desert, because they have neither sanctuary nor use
for it. They can also only farm on rare fertile soil and their yields are small
compared to other people.
