\chapter{Goods}\label{ch:Goods}
This chapter describes the different goods produced and gathered within the
world.

\section{Natures bounty}\label{ch:Goods:Nature}
The world is rich in countless natural resources the tribes can use. They can
be categorized in wild animals, plants and minerals. These are described in
this section.

\subsection{What grows}\label{ch:Goods:Nature:Plants}
The world is home to a wide variety of plants. Plants can be categorized into
plants of special interesting like trees or herbs and abundant plants. Abundant
usually cover large areas and are to abundant to track. \emph{Grass} is a good
example. In the following list all special plants will be listed with the
\emph{fruit} they create continuously and the resources they can be
\emph{harvested} for.

\begin{longtable}{llllll}
	\toprule
	Plant         & Fruit    & Harvest        & Biome             & Growth & Abundance \\
	\midrule
	Herb          & None     & \Gls{Herb}     & Forest            & High   & Low       \\
	Mushroom      & None     & \Gls{Mushroom} & Forest            & Medium & Low       \\
	Fir           & None     & \Gls{Timber}   & Forest            & Low    & High      \\
	Oak           & None     & \Gls{Timber}   & Forest, Grassland & Lowest & Lower     \\
	Apple-tree    & Apple    & \Gls{Timber}   & Forest, Grassland & Lower  & Lower     \\
	Chestnut-tree & Chestnut & \Gls{Timber}   & Forest, Grassland & Lower  & Lower     \\
	\bottomrule
\end{longtable}

\subsection{What grows in abundance}\label{ch:Goods:Nature:Plants:Abundant}
Abundant plants usually cover the entire ground of a Biome, therefore they are
not tracked separately.

\paragraph{Grass}
Grass covers the ground within the forests and the grasslands. Some livestock
needs to graze to live. This can only happen on grassland. The grass grows fast
enough so it is not affected by the grazing.

\subsection{What grows and is searched}\label{ch:Goods:Nature:Plants:Limited}
Most plants are limited in their occurrence. So while they regrows their total
number within an area is limited. The easiest way to understand the concept are
trees. A grove consists of a few trees, once they are felled they are gone.
Mushrooms, herbs and other plants follow a similar principle.

\paragraph{Herbs}
Within the forest medicinal herbs grow. These are can also be used as a food
additives.

\paragraph{Mushromm}
Mushrooms grow within the forest and can be used for medical purposes or as
food.

\paragraph{Fir}
The fir grows within the forest and can be felled for timber.

\paragraph{Oak}
Oak trees dot the forest and grassland and can be felled for timber.

\paragraph{Apple-tree}
Within the grassland and the forest the fruit of apple trees appear, unless
they are cut down for timber.

\paragraph{Chestnut-tree}
The Chestnuts grow in the grassland and the forest and are often consumed by
wild animals, but some of the tribe appreciate them too, others just see
valuable building material in them.

\subsection{What walks}\label{ch:Goods:Nature:Animals}
Within the world there are countless
\hyperref[ch:World:Inhabitants:Animals]{wild animals}. These can be hunted and
butchered for different goods. The animals and the results of their slaughter
are listed here:

\begin{longtable}{cc}
	\toprule
	Animal & Uses                   \\
	\midrule
	Deer   & \Gls{Meat}, \gls{Hide} \\
	Boar   & \Gls{Meat}, \gls{Hide} \\
	\bottomrule
\end{longtable}

\subsection{What swims}\label{ch:Goods:Nature:Sea}
The seas teem with life. The fish within them are caught by the fishers for
sustenance.

\subsection{What stands unmoved}\label{ch:Goods:Nature:Minerals}
Within the worlds above the ground valuable resources often appear agglomerated
within nodes. From these nodes they can be mined. How \emph{hard} they are to
mine and how \emph{abundant} they are is listed here.

\begin{longtable}{ccc}
	\toprule
	Mineral & Hardness & Abundance \\
	\midrule
	Rock    & Low      & Highest   \\
	Copper  & Low      & Low       \\
	Iron    & Higher   & Low       \\
	Gold    & Lower    & Lowest    \\
	Coal    & Lowest   & High      \\
	\bottomrule
\end{longtable}

\paragraph{Rock}
Rock is mined for stone and then used as a building material.

\paragraph{Copper}
Copper is the softest metal used for weapons, tools and armor.

\paragraph{Iron}
Iron is the medium metal for weapons, tools and armor. It can also be refined
into steel.

\paragraph{Gold}
Gold is a rare metal used for luxury items and sometimes buildings.

\paragraph{Coal}
Coal can be mined and used for smelting metals.

\section{Building materials}\label{ch:Goods:Materials}
The different \hyperref[ch:Tribes]{tribes} build with various materials. These
come into different categories like stone and quality levels, so we can have
stones, stone blocks, polished stone blocks and engraved stone blocks. Quality
is always improved by a specialized craftsmen, so to transform a stone into a
stone-block a mason and some time is necessary. To transform boards into
painted boards a painter and paint is necessary. Lower qualities can be
substituted by higher quality, so instead of a stone block a polished or
engraved stone block can be used. During the building process a substituted
material is degraded into lower quality, so if the building is torn down only
the lower quality material can be returned.

Let us now take a look at the different material and quality levels.
\begin{longtable}{ccccc}
	\toprule
	Material & Level 1      & Level 2     & Level 3        & Level 4        \\
	\midrule
	Rock     & Stone        & Stone block & Polished block & Engraved block \\
	Wood     & \Gls{Timber} & Lumber      & Boards         & Painted Boards \\
\end{longtable}

\section{Armory}\label{ch:Goods:Armory}
As the inhabitants of this world are rather war-like, the various implements of
conflict are listed here. In general the quality of the devices depends on the
materials used for them. For a better overview the materials can be found
within this table.

\begin{longtable}{cccc}
	\toprule
	Material   & Armor-Tier & Shield-Tier & Melee-Tier \\
	\midrule
	Leather    & 1          & 0           & None       \\
	Wood       & None       & 1           & 0          \\
	Copper     & 2          & 2           & 2          \\
	Iron       & 3          & 3           & 3          \\
	Steel      & 4          & 4           & 4          \\
	% Extension space for more metals
	Moonsilver & 7          & 6           & 6          \\ \bottomrule
\end{longtable}

\subsection{Weapons}\label{ch:Goods:Armory:Weapons}
This section lists the various weapons and their qualities. They are
categorized according to the \emph{damage} they do to opponents without armor,
the ease with which they \emph{penetrate} armor, the \emph{speed} they permit
their users attacks, the \emph{training} necessary to employ them efficiently
and their \emph{range}.

\begin{longtable}{lllllr}
	\toprule
	Weapon
	       & Damage  & Penetration
	       & Speed   & Training    & Range \\
	\midrule
	Spear
	       & Low     & Medium
	       & Medium  & Low         & 2 m   \\
	Short Sword
	       & Medium  & Medium
	       & Higher  & High        & 1 m   \\
	Battle Axe
	       & Highest & Lower
	       & Medium  & Lower       & 1 m   \\
	Short bow
	       & Medium  & Lower       &
	Medium & High    & 30 m                \\
	\bottomrule
\end{longtable}

Weapons do not only influence the physical battle but also the mental fortitude
of the enemy. To strike \emph{fear} into their enemies and improve their own
\emph{morale} some tribes decorate their weapons- The following list summarizes
the effect of the decorations.

\begin{longtable}{l ll ll ll}
	\toprule
	Weapon
	 & \multicolumn{3}{c}{Morale}
	 & \multicolumn{3}{c}{Fear}
	\\
	 & Plain                      & Simple & Elegant
	 & Plain                      & Simple & Elegant \\
	\midrule
	Spear
	 & Lowest                     & Lower  & Low
	 & Lowest                     & Lower  & Low     \\
	Short Sword
	 & Lower                      & Medium & High
	 & Lowest                     & Lower  & Low     \\
	Battle Axe
	 & Medium                     & High   & Higher
	 & Lower                      & Medium & High    \\
	Short bow
	 & Lower                      & Low    & Medium
	 & Lowest                     & Lowest & Lowest  \\
	\bottomrule
\end{longtable}

\paragraph{Spear}
Spears or sharpened sticks are ancient weapons. A skilled fighter can hit the
weak spots of enemy armor with it. Its ease of production and long range make
it a quite common weapon for the tribes that employ it.

\paragraph{Sword}
Sword or longer knives are very common weapons. Long swords fare better against
lighter armoured opponents, while short swords offer a good compromise for
melee weapons.

\paragraph{Battle axe}
The battle axe is a brute instrument that excels against unarmed opponents. The
ease of use and devastating effect make it the favorite weapon of the
\gls{Vikings}.

\paragraph{Bow}
Bows are acceptable ranged-weapons. They do better against lightly armored
enemies and have an acceptable range.

\subsection{Armor}\label{ch:Goods:Armory:Armor}
To protect them selves against their opponents the tribes fabricate different
forms of armor and defenses. Those are listed here and categorized according to
the protection they offer against \emph{close ranged} weapons, like spears and
swords, the protection they offer against \emph{projectile} weapons like bows.
In addition the \emph{weight} inhibiting the wearers movement and the effect on
the wearers \emph{morale} are listed.

\begin{longtable}{lllll}
	\toprule
	Equipment
	 & Close range & Projectile & Weight  & Morale  \\
	\midrule
	Light body-armor
	 & Medium      & Low        & Low     & Low     \\
	Medium body-armor
	 & Higher      & Medium     & High    & Medium  \\
	Heavy body-armor
	 & Highest     & High       & Highest & High    \\
	\midrule
	Light shield
	 & Lowest      & Low        & Lowest  & Lowest  \\
	Medium shield
	 & Lower       & High       & Lower   & Lower   \\
	Heavy shield
	 & Medium      & Highest    & Low     & Low     \\
	\midrule
	Light helmet
	 & Lowest      & Lowest     & Lowest  & High    \\
	Medium helmet
	 & Lower       & Lowest     & Lower   & Higher  \\
	Heavy helmet
	 & Low         & Lowest     & Low     & Highest \\
	\bottomrule
\end{longtable}

Decorations that are added by some tribes to the pieces of armor have different
effects on the morale of the wearer. No equipment will lower the morale of the
wearer but the influence of the different values of elaboration are listed
here.

\begin{longtable}{llll}
	\toprule
	Category   & Plain  & Simple & Elegant \\
	\midrule
	Body-armor & Lower  & Low    & Medium  \\
	Shield     & Lowest & Lower  & Low     \\
	Helmet     & Lower  & High   & Highest \\
	\bottomrule
\end{longtable}

\paragraph{Body armor}
Body armor is primarily supposed to protect the wearer from harm. It is usually
categorized into light, medium and heavy according to hiw its weight and the
mobility it permits its wearer. The decorated variations are more prestigious,
instilling in the wearers the desire to act braver than they would otherwise.

\paragraph{Shield}
Shields are mostly used to protect against projectile weapons, were they excel.
Like body armor they maybe decorated to improve the warriors morale.

\paragraph{Helmets}
While helmets protect the heads of the wearers. They are often richly decorated
convening prestige and rank. In general a fancy hat means a warrior stays in
battle longer, before they route.
