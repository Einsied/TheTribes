\chapter{The world}\label{ch:World}

The tribes you will play in is settled by different
\hyperref[ch:Tribes]{tribes}. These dwell together with wild animals, great
beasts and precious resources in the forests, jungles, grasslands and deserts
of the world.

To protect themselves from the elements and their opponents most of them chose
to erect great fortified cities and build thriving economies, governed by their
unique approach to the problems of the world and the deities they do or do not
worship.

To lead your chosen tribe to greatness you should learn about it, the other
tribes and the world they inhabit. This document is supposed to support you to
do this.

\section{Biomes}\label{ch:World:Biomes}

Let us begin by talking about the different biomes within the world. These are
more or less hospitable in general and for some tribes specifically. They are
defined by different flora, fauna and resource abundance. A short summary is
provided here.

\begin{longtable}{lllll}
	\toprule
	Biome           & Trees   & Fertility & Animal-Danger & Minerals \\
	\midrule
	\Gls{Grassland} & Low     & Medium    & Lowest        & Lower    \\
	\Gls{Forest}    & Higher  & Medium    & Low           & Lower    \\
	\Gls{Jungle}    & Highest & Highest   & Higher        & Lowest   \\
	\Gls{Desert}    & Lowest  & Lower     & Higher        & Medium   \\
	\Gls{Mountain}  & Lower   & Lowest    & Higher        & Highest  \\
	\bottomrule
\end{longtable}

\subsection{\Glsfmttext{Grassland}}\label{ch:World:Biomes:Grassland}

The \Gls*{Grassland} is the kindest of all biomes. It is fertile for farming,
hosts rather tame wildlife and few trees. Between the rocks some resource veins
might be discovered.

\subsection{\Glsfmttext{Forest}}\label{ch:World:Biomes:Forest}

The \gls*{Forest} is quite similar to the \gls{Grassland}, but hosts some more
hostile wildlife and an abundance of trees. Within it mushrooms and herbs
flourish and a few resource veins can be found.

\subsection{\Glsfmttext{Jungle}}\label{ch:World:Biomes:Jungle}

The \gls*{Jungle} is lush and fertile. Rare in minerals rich in dangerous
animals it is not well suited for most tribes. If the animals can be tamed and
the trees uprooted it is the most fertile biome.

\subsection{\Glsfmttext{Desert}}\label{ch:World:Biomes:Desert}

The endless dry sand and rock of the \gls*{Desert} hosts almost no life. The
animals that roam there are dangerous. The only fertile spots are found along
rivers and oasis. While it is poor land for farming it is excellent for mining,
giving a boost to the tribes that can tame this terrain.

\subsection{\Glsfmttext{Mountain}}\label{ch:World:Biomes:Mountain}

Bare rock bears no fruit and so no tribe can live on the \gls*{Mountain} alone.
Despite that and the more dangerous wildlife the rich ore veins and the
abundance of rock encourage the tribes to build mining outposts within to feed
their growing industry.

\subsection{\Glsfmttext{Sea}}\label{ch:World:Biomes:Sea}

The \gls*{Sea} is not a biome that can be inhabited by the tribes, but its
bounty they can use.

\printglossary[type=biome, title=Biome-Glossary]{}\label{ch:World:BiomesGlossary}

\section{Inhabitants}\label{ch:World:Inhabitants}

The world is inhabited by sapient creatures mostly organized into the various
\hyperref[ch:Tribes]{tribes} and animals. The animals fall into two general
categories \hyperref[ch:World:Inhabitants:Animals]{wild animals} and
\hyperref[ch:World:Inhbitants:Livestock]{livestock}. These will be described in
this section.

\subsection{Wild animals}\label{ch:World:Inhabitants:Animals}

Wild animals roam the different biomes. Some \hyperref[ch:Tribes]{tribes} hunt
them for a variety of \hyperref[ch:Goods:Nature:Animals]{goods}, but their
numbers are small and they replenish slowly, which makes them a finite
resource. Only the \gls{Vikings} can use them for a sustainable economy,
because their \hyperref[ch:Tribes:Vikings:Religion:Forest]{sanctuaries}
replenish them.

\begin{longtable}{llll}
	\toprule
	Animal     & Biome                         & Danger & Speed  \\
	\midrule
	\Gls{Deer} & \Gls{Forest}                  & None   & Higher \\
	\Gls{Boar} & \Gls{Grassland}, \Gls{Forest} & Low    & Medium \\
	\bottomrule
\end{longtable}

\paragraph{Deer}
Deer are forest dwellers. They pose no danger to the tribes and flee, when
spotted or attacked. They are hunted for \gls{Meat} and \glspl{Hide}.

\paragraph{Boars}
Boars dwell on the grassland and the forest. They are hunted for their
\gls{Meat} and \glspl{Hide} and often fight back against the hunters.

\subsection{Livestock}\label{ch:World:Inhabitants:Livestock}

The \hyperref[ch:Tribes]{tribes} keep different animals as livestock to produce
a variety of different \hyperref[label]{goods}. Some of these animals need to
graze and some need to be fed. This list gives a rough summary of all of them.

\subsection{Sapients}\label{ch:World:Inhabitants:Sapients}

Sapients are all sapient creatures. The biggest group are the members of the
\hyperref[ch:Tribes]{tribes}, but it refers to all inhabitants that can use
weapons or equipment.

Those all have two kind of needs, \emph{necessary} needs like \gls{Sustenance}
or \gls{Sleep}, that lead to sever consequences if not fulfilled and
\emph{additional} needs like \gls{Clothing}, which influence their Happiness.
The \emph{necessary} vary among species and the \emph{additional} needs vary
among culture.

Opposed to necessary needs \emph{additional} needs often come with an expected
quality. So while sapients will eat everything for \gls{Sustenance}, they are
more picky for \gls{Dining}. If the \gls{Food} offered is not of sufficient
quality, they will satisfy their need for \gls{Sustenance}, but their need for
\gls{Dining} will stay untouched, leading either to a decrease in happiness or
the consumption of another higher quality \gls{Food}.

While all \emph{additional} needs only influence Happiness by using
\gls{Merchandise} \emph{necessary} needs have different effects:

\begin{longtable}{ll}
	\toprule
	Need             & Effect if unfulfilled \\
	\midrule
	\Gls{Sustenance} & Death                 \\
	\Gls{Sleep}      & Unconsciousness       \\
	\bottomrule
\end{longtable}

As sapients by our definition use \gls{Equipment}, \glspl{Tool}, \gls{Armor}
and \glspl{Weapon} they need to hold them. So sapients have different slots for
these things. Feet for \glspl{Shoe}, hands for \glspl{Weapon} and \glspl{Tool}
and heads and bodies for \gls{Armor}. After these definitions are done we can
now begin th list them.

\subsubsection{Humans}

Humans are the dominant species in the world their \hyperref[ch:tribes]{tribes}
erect great and powerful cities and wage \hyperref[ch:Conflict]{war} against
each others. Here they are categorized as described above.

\begin{longtable}{ll}
	\toprule
	Need             & Strength \\
	\midrule
	\Gls{Sustenance} & Medium   \\
	\Gls{Sleep}      & Medium   \\
	\bottomrule
\end{longtable}

\begin{longtable}{lll}
	\toprule
	Slot         & Amount & Items                                                                   \\
	\midrule
	Pair of feet & 1      & \Glspl{Shoe}                                                            \\
	Head         & 1      & \hyperref[ch:Goods:Armory:Armor]{Helmets}                               \\
	Body         & 1      & \hyperref[ch:Goods:Armory:Armor]{Armor}                                 \\
	Hands        & 2      & \glspl{Tool}, \glspl{Weapon}, \hyperref[ch:Goods:Armory:Armor]{shields} \\
	\bottomrule
\end{longtable}

\printglossary[type=need, title=Need-Glossary]{}\label{ch:World:NeedsGlossary}
\printglossary[type=creature, title=Creature-Glossary]{}\label{ch:World:CreaturesGlossary}
